\chapter{Literature Review}

\section{History of Nuclear Fuel Cycle Simulators}
% What is a nuclear fuel cycle
The \gls{NFC} represents the life cycle of nuclear fuel from initial
extraction, processing, use in reactors, and eventually to 
final disposal.
It is a complex system of facilities and mass flows 
that are combined to meet the goal of providing nuclear energy 
in the 
form of electricity \cite{yacout_modeling_2005}.
An open \gls{NFC} is if used fuel is not reprocessed and a 
closed \gls{NFC} is if used fuel is reprocessed. 
As mentioned previously the \gls{US} runs an open 
\gls{NFC}; other countries, such as France, use a closed \gls{NFC}.

% Purpose of NFCSims 
\glspl{NFCSim} are system analysis tools used to evaluate 
quantitative measures of performance related to the dynamics of 
a \gls{NFC} in both high and low resolution. 
An example of a high-resolution element would be plutonium 
concentration in a single used fuel bundle, while an example 
of a low-resolution element would be total electricity produced. 
The purpose of \glspl{NFCSim} is to better understand the 
dependence between various input parameters and components 
in the \gls{NFC} and the impact of their variations on 
the system's performance. 
The results of \glspl{NFCSim} are being used to guide research 
efforts, advise future design choices, and provide 
decision makers with a transparent tool for evaluating \glspl{FCO} 
to inform big-picture policy decisions \cite{yacout_modeling_2005}.

% Where are NFC codes from 
Historically, national laboratories around the globe have driven 
development and are the primary users of \gls{NFC} simulator tools. 
However, due to propriety access to these tools, universities and 
other non-laboratory organizations have taken to creating their 
own \gls{NFC} simulator tools. 
Table \ref{tab:nfctools} shows a breakdown of all major \glspl{NFCSim}
and the organization(s) associated with them.  

\begin{table}[]
    \centering
    \resizebox{0.7\textwidth}{!}{%
    \begin{tabular}{|l|l|}
    \hline
    \textbf{\gls{NFC} simulator} & \textbf{Organization(s) associated with it}                                    \\ \hline
    \Cyclus \cite{huff_fundamental_2016}                & \begin{tabular}[c]{@{}l@{}}\gls{UW} \\ \gls{UIUC}\end{tabular} \\ \hline
    DYMOND \cite{yacout_modeling_2005},                                & \gls{ANL}                                                                                               \\ \hline
    ORION  \cite{gregg_analysis_2012}                                & \gls{NNL}                                                                                             \\ \hline
    VISION \cite{jacobson_vision:_2006}                                & \gls{INL}                                                                                               \\ \hline
    COSI   \cite{coquelet-pascal_cosi6:_2015}                                &   \gls{CEA}                    \\ \hline
    CLASS  \cite{mouginot_class_2012}                                &  \begin{tabular}[c]{@{}l@{}}\gls{CNRS} \\ \gls{IRSN}\end{tabular}                                      \\ \hline
    DESAE  \cite{tsibulskiy_desae_2006} & \gls{OECD} \\ \hline
    \end{tabular}%
    }
    \caption{Nuclear Fuel Cycle Simulator Tools and their corresponding organizations.}
    \label{tab:nfctools}
    \end{table}

% what is agent, what is fleet
\glspl{NFCSim} are broken into two main groups based on their 
modeling method: fleet-level and agent-level.  
Fleet-based models do not distinguish between discrete facilities 
or materials, instead lumping them into fleets and streams.
The advantages of this method are a simpler code structure and 
lower computational cost. 
Agent-based models treat facilities and materials as discrete 
objects. 
The advantages of this method are more flexible simulation control
and ease of simulating a wide range of scenarios with new 
technologies.  

% why is it good that there are multiple tools 
% benchmark against each other
Efforts have been made towards benchmarking \gls{NFC} simulator 
tools against each other to verify them 
\cite{feng_standardized_2016,guerin_benchmark_2009}. 
These comparison studies assist developers in modifying the
codes to more realistically model the \gls{NFC}. 
Furthermore, by upholding the \gls{NFC} simulator tools to high level 
agreement, 
stakeholders and decision makers can have more confidence in 
prediction results generated by \gls{NFC} simulator tools and trust them 
to inform on potential strategic and policy decisions
\cite{feng_standardized_2016}. 

\section{Transition Scenarios}
In Chapter \ref{chap:1}, the history and motivation of
\gls{NFC} transition scenario research was described.
More detail about transition scenario studies will be given 
in this section. 

The evaluation and screening study identified 40 promising 
\glspl{EG} to represent a comprehensive set of 
\gls{FCO} \cite{wigeland_nuclear_2014}. 
To access the performance of each \gls{EG}, the study
used 9 evaluation criteria: nuclear waste management, 
proliferation risk, nuclear material security risk, 
safety, environmental impact, resource utilization, 
development and deployment risk, institutional issues, and 
financial risk.  
The study concluded that fuel cycles
involving continuous recycling of co-extracted U/Pu or U/TRU in 
fast spectrum critical reactors consistently scored high overall 
performance.
In the study, these fuel cycles were referred to as EG23, EG24, 
EG29 and EG30. 
Table \ref{tab:eg} provides a description of the current 
\gls{US} \gls{EG} and the promising \glspl{EG}. 
These \glspl{EG} were evaluated at an equilibrium state to 
understand the end-state benefits of each \gls{EG}.
Knowing the most promising end state \glspl{EG}, 
the next step is to evaluate and compare the transition process 
from the current EG01 
state to these promising \glspl{EG} \cite{feng_standardized_2016}. 
\begin{table}[]
    \centering
    \caption{Descriptions of EG01 (current) and EG23,24,29,30 (high performing nuclear fuel cycle evaluation groups) described in the evaluation and screening study \cite{wigeland_nuclear_2014}.}
    \label{tab:eg}
        \scriptsize
        \begin{tabularx}{\textwidth}{l|RRR}
            \hline
        \textbf{Fuel Cycle}                                               & \textbf{Open or Closed} & \textbf{Fuel Type}                                                              & \textbf{Reactor Type}                                                                           \\ \hline
        \textbf{\begin{tabular}[c]{@{}l@{}}EG01\\ (current)\end{tabular}} & Open                                                               & Enriched-U                                                                      & Thermal critical                                                                  \\ 
        \textbf{EG23}                                                     & Closed                                                             & \begin{tabular}[c]{@{}l@{}}Recycle of U/Pu \\ with natural-U fuel\end{tabular}  & Fast critical                                                                        \\ 
        \textbf{EG24}                                                     & Closed                                                             & \begin{tabular}[c]{@{}l@{}}Recycle of U/TRU \\ with natural-U fuel\end{tabular} & Fast critical                                                                         \\ 
        \textbf{EG29}                                                     & Closed                                                             & \begin{tabular}[c]{@{}l@{}}Recycle of U/Pu \\ with natural-U fuel\end{tabular}  & \begin{tabular}[c]{@{}l@{}}Fast critical and \\ thermal critical \end{tabular} \\ 
        \textbf{EG30} & Closed                                                             & \begin{tabular}[c]{@{}l@{}}Recycle of U/TRU \\ with natural-U fuel\end{tabular} & \begin{tabular}[c]{@{}l@{}}Fast critical and \\ thermal critical \end{tabular} \\ \hline
    \end{tabularx}
\end{table}


\section{\glspl{NFCSim} Transition Scenario Capabilities}
% what has been done for transition scenarios so far
Both \gls{NFC} simulator tools used in this thesis, \Cyclus and DYMOND,
were verified in a transition scenario benchmarking effort
\cite{feng_standardized_2016,bae_standardized_2019}.
The reference problem used in the benchmark was a simplified 
transition of one hundred 1000-MWe \glspl{LWR} to a fleet 
of 333.3-MWe \gls{SFR} fleet. 
They were found to have excellent agreement with the 
spreadsheet solution and other \gls{NFC} codes.  
This benchmarking effort proved that these \glspl{NFCSim}
are capable of simulating a simple transition scenario. 
However, it acknowledged that there need to be more efforts 
made to model realistic transition scenarios to evaluate the
flexibility of the \glspl{NFCSim} \cite{feng_standardized_2016}.
In-depth descriptions about \Cyclus and DYMOND is provided in 
Chapter \ref{chap:3}.

% brown paper 
% teddy thesis for cyclus capabilties 

\section{\glspl{NFCSim} Sensitivity Analysis}
% SA for NFCs vs SA for NFC transition scenarios 

To enable \glspl{NFCSim} to produce insightful and 
flexible results to inform policy decisions, it is necessary 
to be able to quantify and include all the 
subtleties of each segment of the \gls{NFC} through system analysis 
and sensitivity studies \cite{passerini_systematic_2014}. 
Simulated transition scenarios are intended to predict the future, 
however when implemented in the real world these simulations 
naturally deviate from the optimal.
Therefore, sensitivity analysis studies are necessary to determine 
parameter variation will impact the 
progression and final state of the transition scenario. 

Previous work towards Sensitivity Analysis (SA) and \gls{UQ} of 
\gls{NFC} simulations used these terms interchangeably. 
This is because \gls{UQ} in \glspl{NFCSim} is seen as a design 
uncertainty. 
For example, a pyrochemical reprocessing facility has never been 
built therefore its throughput is viewed as a design parameter 
that can be varied. 
By conducting \gls{SA}/\gls{UQ} on the throughput design 
parameter, the effect on important output parameters can be 
determined. 
Similarly, by conducting studies on a larger set of input 
parameters it is possible to determine which parameters the 
scenario is most sensitive to.
This provides a target of where closer sensitivity study 
should be conducted and additional modeling detail should be added. 
It will also identify which parameters the system is relatively 
insensitive to \cite{noauthor_effects_2017}. 

\gls{SA} is a technique used to determine how 
varying input variable(s) will impact the 
output of a given scenario. 
Many assumptions are made when setting up the simulation scenarios, 
therefore, \gls{SA} will assist in the evaluation of
sensitivity of the output of the scenario to each of these 
assumptions.  
There are three types of sensitivity analysis: one-at-a-time (OAT), 
synergistic and global. 

\subsection{One-at-a-time Sensitivity Analysis}
OAT is basic \gls{SA} technique that focuses on estimating 
the lone effect of one input variable. 
This approach gives the local impact of each variable on the 
output parameters of interest. 
OECD conducted an OAT sensitivity analysis \cite{noauthor_effects_2017} 
on key \gls{NFC} input parameters
and quantified the impacts on the selected output parameters. 
The base scenario used has a duration of 200 years and begins 
with a fleet of \glspl{PWR}, transitioning to \glspl{SFR} while 
maintaining constant electricity production. 
Each parameter was varied independently for three cases: 
the base case, a high case, and a low case (relative to the base 
case). 
The results of these variations on the output parameters 
are expressed in tornado plots and sensitivity tables. 
Figure \ref{fig:oecd-sensitivitytable} shows the sensitivity table
provided by the benchmark that gives an overview of their analysis. 
Figure \ref{fig:oecd-tornado} shows an example tornado plot that represents 
the sensitivity of separated Pu in storage to the various input parameters. 

\begin{figure}[]
	\begin{center}
		\includegraphics[scale=0.55]{./figures/oecd-sensitivitytable.png}
	\end{center}	
		\caption{Sensitivity Table that provides an overview of the sensitivity 
		of each output parameter to the respective input parameters \cite{noauthor_effects_2017}.}
	\label{fig:oecd-sensitivitytable}
\end{figure}

\begin{figure}[]
	\begin{center}
		\includegraphics[scale=0.65]{./figures/oecd-tornado.png}
	\end{center}	
		\caption{Tornado plot showing the sensitivity of separated Pu in 
		storage to relation to each input parameter \cite{noauthor_effects_2017}.}
	\label{fig:oecd-tornado}
\end{figure}

\subsection{Synergistic Sensitivity Analysis}
The Synergistic \gls{SA} technique involves multi-parameter 
input sweeps to view the impact of synergistic 
changes of input variables on specific output variables. 
Synergistic \gls{SA} can be conducted by varying 
two input variables simultaneously and viewing their 
combined impact on each output parameter or a combination 
of weighted output parameters. 
Figure \ref{fig:passerini_payoff} shows an example of this analysis.
Thermal reprocessing and fast reactor technology introduction dates
were varied and an objective payoff surface representing a combination 
of multiple optimization criteria is shown. 
This method of synergistic study informs on the local effects of 
two input variables on the system, however, it fails to inform 
on the global sensitivity of the system. 

\begin{figure}[]
	\begin{center}
		\includegraphics[scale=0.25]{./figures/passerini_payoff.jpg}
	\end{center}	
		\caption{Optimization Surface of the payoff when varying thermal 
		reprocessing and fast reactor technology introduction date
		\cite{passerini_systematic_2014}}
	\label{fig:passerini_payoff}
\end{figure} 

\subsection{Global Sensitivity Analysis}
To fully consider the synergistic effects of
simultaneous variation of the input variables, a variance 
based approach can be used instead \cite{thiolliere_methodology_2018}.
Thiolliere et al conducted a global sensitivity analysis of a 
\gls{NFC} scenario by using Latin Hypercube sampling 
to generate Sobol indices. 
The scenario was a simulation of a simplified PWR-UOX MOX fuel 
fleet. 
Sobol Indices provide the global sensitivity effect of each input 
variable by decomposing the variance of the output into fractions 
that are attributed to inputs or sets of inputs. 
Essentially, it indicates which design parameter have the most 
influence on the response quantities. 
Large Sobol indices signify that variation in that 
input variable is more impactful to the output parameter. 

\subsection{Main Takeaways}
\gls{SA} studies of \glspl{NFC} have previously been used to narrow 
down and compare a wide range of \gls{NFC} scenarios to determine 
the ideal scenario end types. 
The conclusions are that the main trade-off for fuel cycle 
optimization is economics as opposed to
other metrics such as environmental impact, proliferation 
risk \cite{passerini_systematic_2014}.
It was determined that the desired fuel cycle end states 
were EG23, EG24, EG29, and EG30.
These \gls{SA} studies focused on high level input 
parameters such as reactor and reprocessing technologies etc.
However, limited sensitivity studies have been performed to 
evaluate specific transition scenarios that describe the transition 
from the current to desired end states.
The only relevant sensitivity study was conducted by OECD 
\cite{noauthor_effects_2017}, however it was a basic OAT 
\gls{SA}.   
Therefore, synergistic \gls{SA} studies focused on
lower level input parameters such as cooling time, 
date of introduction of reprocessing/reactor 
technologies, ratio of technology types should be conducted to 
understand the nuances of the variation of these low-level parameters. 
Through synergistic \gls{SA}, these transition scenarios can be 
further optimized and used to inform other nuclear research areas 
such as reprocessing facility design etc. 