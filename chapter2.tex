\chapter[chapter 2]{Literature Review}

\section{History of Nuclear Fuel Cycle Simulators}
% What is a nuclear fuel cycle
The \gls{NFC} represents the life cycle of nuclear fuel from initial
extraction, processing, use in reactors, and eventually to 
final disposal.
It is a complex system of facilities and mass flows 
that are combined to meet the goal of providing nuclear energy 
in the 
form of electricity \cite{yacout_modeling_2005}.
An open \gls{NFC} is if used fuel is not reprocessed and a 
closed \gls{NFC} is if used fuel is reprocessed. 
Presently, the \gls{US} has an open \gls{NFC}. 
Whereas, an example of a country that has a 
closed \gls{NFC} is France. 

% Purpose of NFCSims 
\glspl{NFCSim} are system analysis tools used to evaluate 
quantitative measures of performance related to the dynamics of 
a \gls{NFC} on both high and low resolution. 
An example of a high-resolution element is the plutonium 
concentration in a single used fuel bundle, and an example 
of a low-resolution element is total electricity produced. 
The purpose of \glspl{NFCSim} is to better understand the 
dependence between various input parameters and components 
in the \gls{NFC} system and the impact of their variations on 
the system's performance. 
The results of \glspl{NFCSim} are being used to guide research 
efforts, advise future design choices and to provide 
decision makers with a transparent tool for evaluating \glspl{FCO} 
to inform big-picture policy decisions \cite{yacout_modeling_2005}.

% Where are NFC codes from 
Historically, international national laboratories have driven 
development and are the major users of \gls{NFCSim} tools. 
However, due to propriety access to these tools, universities and 
other non-laboratory organizations have taken to creating their 
own tools. 
Table \ref{tab:nfctools} shows a breakdown of the \glspl{NFCSim}
and the organization(s) associated with it. 

\begin{table}[]
    \centering
    \resizebox{0.7\textwidth}{!}{%
    \begin{tabular}{|l|l|}
    \hline
    \textbf{\gls{NFCSim}} & \textbf{Organization(s) associated with it}                                    \\ \hline
    \Cyclus \cite{huff_fundamental_2016}                & \begin{tabular}[c]{@{}l@{}}\gls{UW} \\ \gls{UIUC}\end{tabular} \\ \hline
    DYMOND \cite{yacout_modeling_2005},                                & \gls{ANL}                                                                                               \\ \hline
    ORION  \cite{gregg_analysis_2012}                                & \gls{NNL}                                                                                             \\ \hline
    VISION \cite{jacobson_vision:_2006}                                & \gls{INL}                                                                                               \\ \hline
    COSI   \cite{coquelet-pascal_cosi6:_2015}                                &   \gls{CEA}                    \\ \hline
    CLASS  \cite{mouginot_class_2012}                                &  \begin{tabular}[c]{@{}l@{}}\gls{CNRS} \\ \gls{IRSN}\end{tabular}                                      \\ \hline
    DESAE  & \gls{OECD} \\ \hline
    \end{tabular}%
    }
    \caption{Nuclear Fuel Cycle Simulator Tools and their corresponding organizations.}
    \label{tab:nfctools}
    \end{table}

One of the major distinctions between the various \glspl{NFCSim}
is fleet-level or agent-level modeling of facilities and materials. 
Fleet-based models do not distinguish between discrete facilities 
or materials, but instead lump them together into fleets and streams. 
The advantages of this method is a simpler code structure and 
lower computational cost. 
Agent-based models treats facilities and materials as discrete 
objects. 
The advantages of this method are more flexible simulation control
and ease of simulating a wide range of scenarios with new 
technologies. 
% what is agent, what is fleet 

% why is it good that there are multiple tools 
% benchmark against each other
Efforts have been made towards benchmarking \gls{NFCSim} 
tools against each other to verify them 
\cite{feng_standardized_2016,guerin_benchmark_2009}. 
These comparison studies assist developers in modifying the
codes to more realistically model the \gls{NFC}. 
Also, by upholding the \gls{NFCSim} tools to high level agreement, 
stakeholders and decision makers can have more confidence in 
prediction results generated by \gls{NFCSim} tools and trust them 
to inform on potential strategic and policy decisions
\cite{feng_standardized_2016}. 

% Agent based vs Fleet based 

\section{Transition Scenarios}
In Chapter \ref{chap:1}, the history and motivation of
\gls{NFC} transition scenario research was described.
More detail about transition scenario studies will be given 
in this section. 

The evaluation and screening study identified 40 promising 
\glspl{EG} to represent a comprehensive set of 
\gls{FCO} \cite{wigeland_nuclear_2014}. 
To access the performance of each \gls{EG}, the study
used 9 evaluation criteria: nuclear waste management, 
proliferation risk, nuclear material security risk, 
safety, environmental impact, resource utilization, 
development and deployment risk, institutional issues, and 
financial risk.  
The conclusion of the study is that fuel cycles
involving continuous recycling of co-extracted U/Pu or U/TRU in 
fast spectrum critical reactors consistently scored high overall 
performance.
In the study, these fuel cycles were referred to as EG23, EG24, 
EG29 and EG30. 
Table \ref{tab:eg} provides a description of the current 
\gls{US} \gls{EG} and the promising \glspl{EG}. 
These \glspl{EG} were evaluated at an equilibrium state to 
understand the end-state benefits of each evaluation group (EG).
Knowing the most promising end state \glspl{EG}, 
the next step is to evaluate and compare the transition process 
from the current EG01 
state to the these \glspl{EG} \cite{feng_standardized_2016}. 

\begin{table}[]
	\centering
    \begin{tabular}{|l|l|}
        \hline
        Fuel Cycle & Description                                                                                                                                 \\ \hline
        EG01 (current)      & \begin{tabular}[c]{@{}l@{}}Once-through using enriched-U fuel in \\ thermal critical reactors.\end{tabular}                                 \\ \hline
        EG23       & \begin{tabular}[c]{@{}l@{}}Continuous recycle of U/Pu with new\\ natural-U fuel in fast critical reactors.\end{tabular}                     \\ \hline
        EG24       & \begin{tabular}[c]{@{}l@{}}Continuous recycle of U/TRU with new\\ natural-U fuel in fast critical reactors.\end{tabular}                    \\ \hline
        EG29       & \begin{tabular}[c]{@{}l@{}}Continuous recycle of U/Pu with new\\ natural-U fuel in both fast and thermal\\ critical reactors.\end{tabular}  \\ \hline
        EG30       & \begin{tabular}[c]{@{}l@{}}Continuous recycle of U/TRU with new\\ natural-U fuel in both fast and thermal\\ critical reactors.\end{tabular} \\ \hline
    \end{tabular}
    \caption{Descriptions of the current and other high performing nuclear fuel cycle evaluation groups described in the evaluation and screening study \cite{wigeland_nuclear_2014}.}
    \label{tab:eg}
\end{table}


\section{\glspl{NFCSim} Transition Scenario Capabilities}
% what has been done for transition scenarios so far
Both \gls{NFCSim} tools used in this thesis, \Cyclus and Dymond,
were verified by a benchmarking effort for use of 
\gls{NFCSim} tools to conduct transition scenario analyses
\cite{feng_standardized_2016,bae_standardized_2019}.
The reference problem used in the benchmark was a simplified 
transition of one hundred 1000-MWe \glspl{LWR} to a fleet 
of 333.3-MWe \gls{SFR} fleet. 
They were found to have excellent agreement with the 
spreadsheet solution and other \gls{NFC} codes.  
This benchmarking effort proved that these \glspl{NFCSim}
are capable of simulating a simple transition scenario. 
However, it is acknowledged that there needs to be more efforts 
to model realistic transition scenarios to evaluate the
flexibility of the \glspl{NFCSim} \cite{feng_standardized_2016}.


\section{\glspl{NFC} \gls{SA}}