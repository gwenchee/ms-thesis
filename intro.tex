\chapter[Introduction]{Introduction}
\label{chap:1}

\section{Motivation}
The impact of climate change on natural and human systems 
is increasingly apparent \cite{noauthor_climate_2018}.
Increases in global average 
surface temperatures, sea levels, and larger climate extremes
are a few consequences brought on by elevated Greenhouse Gas (GHG) 
concentrations \cite{noauthor_climate_2018}.
Energy use and production contribute to 
two-thirds of the total Green House Gas (GHG) 
emissions \cite{noauthor_climate_2018}.
Furthermore, as the human population increases and previously 
under-developed nations rapidly urbanize, 
global energy demand is forecasted to increase.  
Energy generation technology selection 
profoundly impacts climate change via growing energy demand. 
Large scale nuclear power plant deployment has significant 
potential to reduce GHG production due to their low 
carbon emissions \cite{noauthor_climate_2018}.  

However, the nuclear power industry is facing four major challenges 
that hinder large scale nuclear power deployment: 
cost, safety, proliferation, and used nuclear fuel 
\cite{massachusetts_institute_of_technology_future_2003}. 
Nuclear power has high capital costs, an unresolved 
long-term used nuclear fuel management strategy, perceived adverse safety, 
environmental, and health effects, and increases the nuclear proliferation risk
\cite{massachusetts_institute_of_technology_future_2003}. 
The nuclear power industry must overcome these four challenges 
to ensure nuclear energy technology's continued global use and expansion.

% Evaluation Screening Study 
The present \gls{NFC} in the \gls{US} is a once-through cycle 
in which fabricated nuclear fuel is used once and placed into 
storage to await disposal. 
This once-through fuel cycle is associated with the four
challenges described above.
To overcome these challenges, the Office of Nuclear Energy's
\gls{FCO} Campaign led an evaluation 
and screening study of a comprehensive set of nuclear \gls{FCO} 
to identify \glspl{FCO} with the potential to substantially 
improve the nuclear fuel cycle in the four challenge areas
\cite{wigeland_nuclear_2014}. 

The study concluded that fuel cycles with continuous recycling
of co-extracted U/Pu or U/TRU in fast spectrum critical reactors
consistently scored high overall performance in the following 
categories: used nuclear fuel management, environmental impact, 
and resource utilization. 
The evaluation and screening study assumed
the nuclear energy systems were at equilibrium to understand each \gls{EG}'s
end-state benefits \cite{feng_standardized_2016}. 
Therefore, evaluation of the transition from the current 
once-through fuel cycle to these promising 
future end-states \cite{feng_standardized_2016} 
is the logical next step, propelling this
nuclear fuel cycle transition scenario analysis research. 

\gls{NFC} simulation tools must automate the transition scenario simulation 
setup to successfully model a time-dependent transition scenario. 
Many existing \gls{NFC} simulator tools have conducted 
transition scenario analyses 
\cite{feng_standardized_2016,bae_standardized_2019,coquelet-pascal_cosi6:_2015}
and faced challenges stemming from the vast input parameter
sample space.
Since many of these input parameters are coupled, it is 
a tedious process to use trial and error to manually find a balance 
between various input parameters to set up a successful transition 
scenario. 
We define a successful transition scenario simulation as one that 
has a minimal power undersupply, minimal undersupply, 
and oversupply of all commodities. 
 
In reality, the real transition process inevitably diverges
from the modeled transition scenario. 
It is insufficient to set up only one transition scenario to model 
nuclear power's future projections.
Therefore, it is imperative to conduct \gls{SA} to understand 
the effect of variation in input parameters on 
performance metrics. 

\section{Objectives}
This thesis' objectives were developed based on the difficulties 
\gls{NFC} simulators face when modeling transition scenarios 
and conducting sensitivity analysis in the context of those scenarios.
Accordingly, we aim to 
(1) develop a capability in \Cyclus to ease the setup of 
transition scenarios, 
(2) develop sensitivity analysis capabilities in \Cyclus and DYMOND, 
(3) demonstrate \Cyclus transition scenario setup using the 
developed capability,
(4) use \Cyclus and DYMOND to conduct sensitivity studies
, and
(5) compare \Cyclus' and DYMOND's capabilities in conducting sensitivity 
analysis. 
