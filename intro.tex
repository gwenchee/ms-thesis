\chapter[Introduction]{Introduction}

\section{Background and motivation}
[Why we want to transition]
The current fleet of \glspl{PWR} are old. 

An evaluation and screening study of a comprehensive set of nuclear 
\gls{FCO} \cite{wigeland_nuclear_2014} was conducted to assess 
for performance improvements compared the existing once-through 
fuel cycle (EG01) in the \gls{US} across a wide range of criteria. 
It was found that fuel cycles that consistently scored high 
overall performance involved continuous recycling
of co-extracted U/Pu or U/TRU in fast spectrum critical reactors. 
The evaluation and screening study assumed that 
the nuclear energy system was at an equilibrium to understand 
the end-state benefits of each evaluation group (EG). 
Based on the results from the study, the next step is 
to understand and evaluate the transition from the initial EG01
state to these promising future end-states 
\cite{feng_standardized_2016}. 

This propelled research towards nuclear fuel cycle transition 
scenario analysis. 
To successfully conduct analysis of the time-dependent transition
analyses, it is necessary to develop \gls{NFCSim} tools to  
automate setting up of transition scenarios. 
Many fuel cycle simulator tools were improved to have 
the capabilities conduct these analyses, such as 
Dymond, Cyclus, ORION... 
[add citations and list]  
The main issues faced by these tools for conducting these transition 
scenario analyses stem from the large sample space of input 
parameters for nuclear fuel cycle scenarios.
Since many of these input parameters are interdependent, it ends
up being a tedious process for the user to manually find a balance 
between various input parameters to set up a successful transition 
scenario. 

Furthermore, since \gls{NFC} simulations are complex systems with 
many input and output variables. 
It is insufficient to just set up one transition scenario to model 
the future projections of nuclear power. 
In reality, despite having an idea of how this one transition 
scenario fares, the real transition process will diverge from it. 
Therefore, it is important to conduct \gls{SA} and optimization 
to understand the effects of variation in input parameters on 
important output parameters. 

\section{Objectives}
Knowing the difficulties faced by \gls{NFCSim}s to model 
transition scenarios and conduct sensitivity analysis, 
the objectives of this thesis are to: 
(1) Develop a capability in \Cyclus to ease setting up of transition 
scenarios, 
(2) Compare \Cyclus and Dymond's capabilities in setting up 
transition scenarios, and 
(3) Compare \Cyclus and Dymond's capabilities in conducting \gls{SA}. 