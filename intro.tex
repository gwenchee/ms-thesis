\chapter[Introduction]{Introduction}
\label{chap:1}

\section{Motivation}
%[Climate Change, carbon free energy, need nuclear]

The impact of climate change on natural and human systems has 
become increasingly apparent.
Climate change manifests as increases in global average 
surface temperatures, sea levels, and larger climate extremes.
These are consequences of elevated Greenhouse Gas (GHG) 
concentrations. 
Two-thirds of total GHG emissions are from the production 
and use of energy \cite{noauthor_climate_2018}. 
With an increasing human population and rapid urbanization 
of previously under-developed nations, global energy demand is 
forecasted to increase. 
The impact of the growing energy demand on climate change 
will heavily depend on the 
types of power generation technologies used. 
Large scale deployment of nuclear power plants has significant 
potential to reduce GHG production due to their low 
carbon emissions \cite{noauthor_climate_2018}.  

%[Challenges facing large-scale nuclear power deployment]

However, there are four major problems facing the nuclear power 
industry
that challenge the large scale deployment of nuclear power 
: cost, safety, proliferation, and waste 
\cite{massachusetts_institute_of_technology_future_2003}. 
Nuclear power has high overall lifetime costs and brings 
with it the risk of nuclear proliferation. 
It also has perceived adverse safety, environmental, and health 
effects, and an unresolved long-term nuclear waste management 
strategy \cite{massachusetts_institute_of_technology_future_2003}. 
The nuclear power industry must overcome these four challenges 
to ensure continued global use and expansion 
of nuclear energy technology. 

% Evaluation Screening Study 
The present \gls{NFC} in the \gls{US} is a once-through cycle 
in which fabricated nuclear fuel is used once and then placed into 
storage awaiting disposal. 
The challenges described above are associated with this 
once-through fuel cycle. 
Therefore, to overcome and mitigate these challenges the 
\gls{DOE} chartered an evaluation and screening study 
of a comprehensive set of nuclear \gls{FCO} 
to identify \glspl{FCO} with potential 
to provide substantial improvements in the four challenge areas
compared to the current 
\gls{NFC} in the \gls{US} \cite{wigeland_nuclear_2014}. 

The study found that fuel cycles that involved continuous recycling
of co-extracted U/Pu or U/TRU in fast spectrum critical reactors
consistently scored high overall performance in the following 
categories: nuclear waste management, environmental impact, 
and resource utilization. 
The study assumed that 
these nuclear energy systems were at an equilibrium to understand 
the end-state benefits of each evaluation group (EG). 
Therefore, based on the results from this study, the next step is 
to understand and evaluate the transition from the current 
once-through fuel cycle to these promising 
future end-states \cite{feng_standardized_2016}. 
This propelled research towards nuclear fuel cycle transition 
scenario analysis. 

\subsection{Transition Scenarios}
To successfully conduct analysis of the time-dependent transition
scenarios, it is necessary to develop \gls{NFCSim} tools to  
automate the setup of transition scenarios. 
Many existing fuel cycle simulator tools are used to conduct 
these analyses and face challenges stemming from the large 
number and sample space of input 
parameters for nuclear fuel cycle transition scenarios.
Since many of these input parameters are interdependent, it ends
up being a tedious process for the user to manually find a balance 
between various input parameters to set up a successful transition 
scenario. 

Furthermore, since \gls{NFC} simulations are complex systems with 
many input and output variables. 
It is insufficient to just set up one transition scenario to model 
the future projections of nuclear power. 
In reality, the real transition process will 
inevitably diverge from the modeled transition scenario. 
Therefore, it is important to conduct \gls{SA} to understand 
the effects of variation in input parameters on 
important output parameters. 

\section{Objectives}
Two \gls{NFCSim} tools, \Cyclus and DYMOND, will be evaluated on 
their capabilities to setup successful transition scenarios. 
The \Cyclus tool will be enhanced to ease setting up of 
transition scenarios. 
Knowing the difficulties faced by \glspl{NFCSim} to model 
transition scenarios and conduct sensitivity analysis, 
the objectives of this thesis are to: 
(1) develop a capability in \Cyclus to ease the setup of 
transition scenarios, 
(2) evaluate and compare \Cyclus and 
DYMOND based on their capabilities to setup 
successful transition scenarios,
(3) Develop capabilities in \Cyclus and DYMOND to conduct 
sensitivity analysis,
(4) Compare \Cyclus and DYMOND's capabilities in conducting \gls{SA}. 