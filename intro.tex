\chapter[Introduction]{Introduction}

\section{Background and motivation}
\subsection{Why is Nuclear Waste Disposal research important?}
Implementation of a nuclear waste disposal plan and minimizing the 
cost of the nuclear fuel cycle are crucial to the future use of 
nuclear power 
\cite{massachusetts_institute_of_technology_future_2003}. 
If the U.S. nuclear industry does not find an effective and safe 
plan to manage the waste, the nuclear industry will continue facing 
political and social opposition. 

[**More]

\subsection{Possible Methods of Waste disposal}
Currently, all of the commercial nuclear waste in the \gls{US} are 
scattered across the country at each reactor site [** Reference]. 
There has been much debate on what to do with the nuclear waste in 
the long term. 
There has been much debate between an open fuel cycle and a closed 
fuel cycle. 

[**Describe Open Fuel Cycle]

[**Describe Closed Fuel Cycle, why does France use it?]

[**Compare them -- advantages and disadvantages of each]

[**Describe why an open fuel cycle is more feasible in the US]
It was determined in the large multi-disciplinary study of the 
future of nuclear power that a once-through fuel cycle is more 
economic and proliferation-resistant than a closed fuel cycle that
makes use of reprocessing technology
\cite{massachusetts_institute_of_technology_future_2003}.

[**Waste Repository is necessary no matter what because every 
fuel cycle will have non-reprocessable waste]

In this work, the expectation is that the chosen fuel cycle is a 
once through fuel cycle and the method of long term disposal of 
spent nuclear fuel (SNF) will be a deep geologic repository. 

\subsection{Previous Work towards repository modeling}
Previous work towards the wicked problem of getting spent nuclear fuel from reactor 
sites to a final waste repository focuses on how different waste acceptance strategies 
impact economic expenditure \cite{nesbit_proposed_2015}, pre-emplacement 
surface storage time, waste package size, and repository 
footprint \cite{greenberg_application_2012}. 
There has also been efforts to holistically evaluate the entire system to consolidate 
how each factor impact the cost and safety of moving SNF from 
reactor sites to the final waste repository \cite{nutt_waste_2015}.
Previous work in studying repository loading have used spent fuel assemblies 
that have an average burn up composition \cite{johnson_optimizing_2016} 
to evaluate the heat load in the repository \cite{greenberg_application_2012}. 

\section{Objectives}
The objective of this thesis is to: 
\begin{itemize}
    \item Create a \Cyclus spent fuel conditioning model that
    packages spent fuel bundles into packages which have
    user-defined properties.
    \item Create a \Cyclus interim storage facility that gives 
    waste canisters to the repository facility to emplace in the 
    order of a specific waste acceptance strategy. 
    \item Create a \Cyclus medium-fidelity repository model that
    accepts and emplaces canisters that results in the repository 
    remaining below the thermal limit of the host geologic media.
    It should also give accurate time and spatial dependent 
    temperature values in the repository. 
    \item Use U.S. historical SNF inventory data 
    \cite{peterson_unf-st&dards_2017} in various simulations
    that model different loading strategies for moving 
    \gls{SNF} from reactor sites to a final waste repository.
\end{itemize}


\section{Methods}
Explain \Cyclus, \Cycamore, \Cyder, PyNE etc.  

\subsection{Fuel Cycle Modeling}
Ever since the conception of nuclear power, there existed the need 
for a accurate model of the nuclear fuel cycle. 

[** Importance of Nuclear Fuel Cycle Modeling]

[** Export control issues with current codes]
There has been efforts globally towards designing all encompassing 
nuclear fuel cycle simulation codes. 
This is also true in the United States where almost each national 
laboratory in the country has a nuclear fuel cycle code. 
The trouble is that codes that are created in national laboratories 
face export control. 
Therefore, it is difficult for global efforts towards producing 
an excellent fuel cycle code if there is difficulties sharing 
coding practices and techniques used for each code. 

This led to the development of an open source fuel cycle simulation 
code \Cyclus. 

[** Take from fundamentals, reason for Cyclus and why it is good, 
\Cyclus notion of agents and anyone can contribute, \Cyclus 
community maintained archetypes, and other ones (Bright-lite etc.)]

