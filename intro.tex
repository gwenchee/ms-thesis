\chapter[Introduction]{Introduction}
\label{chap:1}

\section{Motivation}
%[Climate Change, carbon free energy, need nuclear]

The impact of climate change on natural and human systems have 
become increasingly apparent.
This manifests in the increase of global average surface 
temperature, sea levels, and changes in climates extremes.
These are consequences of elevated \gls{GHG} concentrations. 
Two-thirds of total \gls{GHG} emissions are from the production 
and use of energy \cite{noauthor_climate_2018}. 
With the increasing human population and urbanizing of less 
developed countries, global energy demand is forecasted to 
increase. 
The impact of the growing demand for energy on climate change 
will depend on the 
types of power generation technologies used. 
Large scale deployment of nuclear power plants has significant 
potential to reduce \gls{GHG} production due to its low 
carbon emissions \cite{noauthor_climate_2018}.  

%[Challenges facing large-scale nuclear power deployment]

However, there are four major problems facing the nuclear power 
industry
that challenge the large scale deployment of nuclear power 
: cost, safety, proliferation, and waste 
\cite{massachusetts_institute_of_technology_future_2003}. 
Nuclear power has high overall lifetime costs and use of it comes 
with the risk of nuclear proliferation. 
It also has perceived adverse safety, environmental, and health 
effects, and an unresolved long-term nuclear waste management 
strategy \cite{massachusetts_institute_of_technology_future_2003}. 
The nuclear power industry must overcome these four challenges 
to ensure continued global use and expansion 
of nuclear energy technology. 

% Evaluation Screening Study 
The present \gls{NFC} in the \gls{US} is a once-through cycle 
in which fabricated nuclear fuel is only used once and not recycled. 
The challenges described above are associated with this 
once-through fuel cycle. 
Therefore, to overcome and mitigate these challenges, the 
\gls{DOE} chartered an evaluation and screening study 
of a comprehensive set of nuclear \gls{FCO} 
to identify \glspl{FCO} with potential 
to provide substantial improvements in the four challenge areas
compared to the current 
\gls{NFC} in the \gls{US} \cite{wigeland_nuclear_2014}. 

The study found that fuel cycles that involved continuous recycling
of co-extracted U/Pu or U/TRU in fast spectrum critical reactors
consistently scored high overall performance. 
The study assumed that 
these nuclear energy systems were at an equilibrium to understand 
the end-state benefits of each evaluation group (EG). 
Therefore, based on the results from this study, the next step is 
to understand and evaluate the transition from the current 
once-through fuel cycle to these promising 
future end-states \cite{feng_standardized_2016}. 
This propelled research towards nuclear fuel cycle transition 
scenario analysis. 

\subsection{Transition Scenarios}
To successfully conduct analysis of the time-dependent transition
analyses, it is necessary to develop \gls{NFCSim} tools to  
automate setting up of transition scenarios. 
Many existing fuel cycle simulator tools are used to conduct 
these analyses and face challenges stemming from the large sample 
space of input 
parameters for nuclear fuel cycle transition scenarios.
Since many of these input parameters are interdependent, it ends
up being a tedious process for the user to manually find a balance 
between various input parameters to set up a successful transition 
scenario. 

Furthermore, since \gls{NFC} simulations are complex systems with 
many input and output variables. 
It is insufficient to just set up one transition scenario to model 
the future projections of nuclear power. 
In reality, the real transition process will 
inevitably diverge from the modeled transition scenario. 
Therefore, it is important to conduct \gls{SA} to understand 
the effects of variation in input parameters on 
important output parameters. 

\section{Objectives}
In this thesis, two \gls{NFCSim} tools, \Cyclus and 
Dymond will be evaluated on their capabilities of setting up 
successful transition scenarios. 
The \Cyclus tool will be enhanced to ease setting up of 
transition scenarios. 
And finally,   
Knowing the difficulties faced by \glspl{NFCSim} to model 
transition scenarios and conduct sensitivity analysis, 
the objectives of this thesis are to: 
(1) Evaluate two \gls{NFCSim} tools, \Cyclus and 
Dymond on their capabilities of setting up 
successful transition scenarios. 
(2) Develop a capability in \Cyclus to ease setting up of transition 
scenarios, 
(3) Compare \Cyclus and Dymond's capabilities in setting up 
transition scenarios, and 
(4) Compare \Cyclus and Dymond's capabilities in conducting \gls{SA}. 