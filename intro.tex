\chapter[Introduction]{Introduction}
\label{chap:1}

\section{Motivation}
%[Climate Change, carbon free energy, need nuclear]

The impact of climate change on natural and human systems 
is increasingly apparent.
Increases in global average 
surface temperatures, sea levels, and larger climate extremes
are a few of the consequences of elevated Greenhouse Gas (GHG) 
concentrations.
The production and use of energy contribute to 
two-thirds of the total Green House Gas (GHG) 
emissions \cite{noauthor_climate_2018}.
Furthermore, as the human population increases, and previously 
under-developed nations urbanize rapidly, 
global energy demand is forecasted to increase.  
The types of power generation technologies used 
profoundly impacts the effects of growing energy demand 
on climate change.  
Large scale deployment of nuclear power plants has significant 
potential to reduce GHG production due to their low 
carbon emissions \cite{noauthor_climate_2018}.  

However, the nuclear power industry is facing four major challenges 
of large scale nuclear power deployment: 
cost, safety, proliferation, and waste 
\cite{massachusetts_institute_of_technology_future_2003}. 
Nuclear power has high overall lifetime costs and increases 
risks of nuclear proliferation. 
There is also an unresolved long-term nuclear waste management 
strategy and perceived adverse safety, environmental, and health 
effects \cite{massachusetts_institute_of_technology_future_2003}. 
The nuclear power industry must overcome these four challenges 
to ensure continued global use and expansion 
of nuclear energy technology. 

% Evaluation Screening Study 
The present \gls{NFC} in the \gls{US} is a once-through cycle 
in which fabricated nuclear fuel is used once and placed into 
storage to await disposal. 
This once-through fuel cycle is associated with the four
challenges described above.
To overcome these challenges, the \gls{DOE} chartered an evaluation 
and screening study of a comprehensive set of nuclear \gls{FCO} 
to identify \glspl{FCO} with potential 
to provide substantial improvements in the four challenge areas
compared to the current 
\gls{NFC} in the \gls{US} \cite{wigeland_nuclear_2014}. 

The study concluded that fuel cycles with continuous recycling
of co-extracted U/Pu or U/TRU in fast spectrum critical reactors
consistently scored high overall performance in the following 
categories: nuclear waste management, environmental impact, 
and resource utilization. 
The evaluation and screening study assumed
the nuclear energy systems were at equilibrium to understand 
the end-state benefits of each \gls{EG} \cite{feng_standardized_2016}. 
Therefore, evaluation of the transition from the current 
once-through fuel cycle to these promising 
future end-states \cite{feng_standardized_2016} 
is the logical next step, propelling 
nuclear fuel cycle transition 
scenario analysis research. 

\subsection{Transition Scenarios}
\gls{NFC} simulator tools must automate the transition scenario simulation 
setup to successfully model a time-dependent transition scenario. 
Many existing \gls{NFC} simulator tools have conducted 
transition scenario analyses 
\cite{feng_standardized_2016,bae_standardized_2019,coquelet-pascal_cosi6:_2015}
and faced challenges stemming from the vast input parameter
sample space.
Since many of these input parameters are interdependent, it is 
a tedious process to use trial and error to manually find a balance 
between various input parameters to set up a successful transition 
scenario. 
A successful transition scenario is a simulation in which there 
is minimal power undersupply.
 
In reality, the real transition process inevitably diverges
from the modeled transition scenario. 
It is insufficient to set up only one transition scenario to model 
the future projections of nuclear power.
Therefore, it is imperative to conduct \gls{SA} to understand 
the effects of variation in input parameters on 
relevant output parameters. 

\section{Objectives}
The objectives of this thesis were developed based on the difficulties 
that NFC simulators face when modeling transition scenarios 
and conducting sensitivity analysis.
The objectives of this thesis are to 
(1) develop a capability in \Cyclus to ease the setup of 
transition scenarios, 
(2) develop capabilities in \Cyclus and DYMOND to conduct 
sensitivity analysis,
(3) demonstrate setting up of \Cyclus transition scenarios using the 
capability developed,
(4) use \Cyclus and DYMOND to conduct sensitivity analysis studies
, and
(5) compare \Cyclus and DYMOND's capabilities in conducting \gls{SA}. 
