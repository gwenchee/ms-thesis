\chapter{Methods}
In this section, the \glspl{NFCSim}, \Cyclus and Dymond, will 
be described and evaluated on their capability of setting up 
successful transition scenarios. 

\section{\Cyclus}
\Cyclus is an agent-based nuclear fuel cycle simulation framework 
\cite{huff_fundamental_2016}. 
In \Cyclus, each entity (i.e. Region, Institution, or Facility) in the 
fuel cycle is an agent. 
Region agents represent geographical or political areas that institution
and facility agents can be grouped into. 
Institution agents control the 
deployment and decommission of facility agents 
and represents legal operating organizations such as a 
utility, government, etc. \cite{huff_fundamental_2016}. 
Facility agents represent nuclear fuel cycle facilities. 
\Cycamore \cite{carlsen_cycamore_2014}
provides facility agents to represent process physics of various 
components in the nuclear fuel cycle (e.g. mine, fuel enrichment 
facility, reactor). 

In 2016, there was a push to understand and evaluate the 
transition from the initial EG01 state to promising future 
end-states \cite{feng_standardized_2016}.
Previously in \Cyclus, reactor facilities are automatically 
deployed to meet a user-defined power demand. 
However, it is up to the user to define a deployment scheme of 
supporting facilities to ensure that there is no gap in the supply 
chain that results in idle reactor capacity. 
Therefore, previously in \Cyclus, some users choose to set 
support facilities to have an infinite capacity to avoid this 
issue, but this is an inaccurate representation of reality. 
It is straightforward to manually determine a deployment scheme 
for a once-through fuel cycle, however, it is difficult to effectively 
implement for complex closed fuel cycle scenarios.  
To successfully conduct analysis of the time-dependent transition
analyses, it is necessary to develop \gls{NFCSim} tools to  
automate setting up of transition scenarios. 
Therefore, Demand-Driven Cycamore Archetypes project
(NEUP-FY16-10512) was initiated to develop demand-driven deployment 
capabilities in \Cyclus.

This capability is added as a \Cyclus Institution
agent that deploys facilities to meet the front-end and back-end 
fuel cycle demands based on a user-defined commodity demand. 
This demand-driven deployment capability is called 
\deploy. 

\subsection{\deploy}
In a \Cyclus \gls{NFC} simulation, at every timestep, \deploy 
predicts supply and demand of each commodity for the next time 
step. 
If there is an undersupply of any commodity based 
on the predicted values, \deploy deploys facilities to meet 
the predicted demand.  
Figure \ref{fig:flow} shows the logic flow of \deploy 
at every timestep. 

\begin{figure}[H]
	\centering
    \begin{tikzpicture}[node distance=2.5cm]
    \tikzstyle{every node}=[font=\large]
	\node (Start) [bblock] {\textbf{Start of timestep ($t$).}};
	\node (Predict) [bblock, below of=Start] {\textbf{Calculate \\ $D_p(t+1)$ and $S_p(t+1)$ for a commodity}};
	\node (IsThere) [oblock, below of=Predict]{\textbf{$U(t+1) = S_p(t+1)-D_p(t+1)$}};
	\node (Deploy) [bblock, below of=IsThere, xshift = -3.5cm]{\textbf{Deployment of facility}};
    \node (NoDeploy) [bblock, right of=Deploy, xshift = 3.5cm]{\textbf{No Deployment} };
    \node (All) [oblock, below of=Deploy, xshift = 3.5cm] {\textbf{Is this done for all commodities?}};
    \node (End) [bblock, below of=All] {\textbf{Proceed to next timestep.}};
	
	\draw [arrow] (Start) -- (Predict); 
	\draw [arrow] (Predict) -- (IsThere);
    \draw [arrow] (IsThere) -- node[anchor=east] {$U(t+1) <$ buffer} (Deploy);
    \draw [arrow] (IsThere) -- node[anchor=west] {$U(t+1) \geq$ buffer} (NoDeploy);
    \draw [arrow] (Deploy) -- (All);
    \draw [arrow] (NoDeploy) -- (All);
    \draw [arrow] (All) -- node[anchor=west] {yes} (End);
    \draw [arrow] (All) -- ([shift={(-3.9cm,0.7cm)}]All.south west)-- node[anchor=east] {no} ([shift={(-3.9cm,-1cm)}]Predict.north west)--(Predict);
    \draw [arrow] (End) |-([shift={(3cm,-0.5cm)}]End.south east)-- ([shift={(3cm,0.5cm)}]Start.north east)-|(Start);
    \end{tikzpicture}
    \label{fig:flow}
    \caption{\deploy logic flow at every timestep in \Cyclus \cite{chee_demonstration_2019}.}
\end{figure}

\deploy's overall objective is to ensure that there is no 
undersupply of power. 
The sub-objectives are : (1) to minimize the number of time 
steps of undersupply or under capacity of any 
commodity, (2): to minimize excessive oversupply of all commodities.
This is a reflection of reality in which it is important to 
never have an undersupply of power on the grid by ensuring power 
plants are never undersupplied of fuel, while not 
having excessive over supply resulting in a burden to store unused 
supplies. 
One of the key issues that \gls{NFCSim}s face is that despite
sufficient installed reactor capacity to meet the power 
demand, there is insufficient supply of fabricated/reprocessed 
fuel at certain timesteps, resulting in idle capacity.  

\subsubsection{Structure}
%Description of front end and back end of fuel cycle 
%Demand Driven vs. Supply Driven 
In \deploy, two different institutions were implemented for 
front-end and back-end fuel cycle facilities: 
\texttt{DemandDrivenDeploymentInst} and 
\texttt{SupplyDriven} 
\noindent
\texttt{DeploymentInst} respectively. 
This distinction was made because front-end facilities 
are deployed to meet demand for the commodity they produce. 
Whereas, back-end facility are deployed to meet supply for the 
commodity they provide capacity for. 
For example, for front end facilities, a reactor facility 
demands fuel and \texttt{DemandDrivenDeploymentInst} 
triggers deployment of fuel fabrication facilities to create 
supply meeting demand for fuel to prevent undersupply. 
For back end facilities, the reactor generates spent fuel and 
\texttt{SupplyDrivenDeploymentInst} triggers deployment of 
waste storage facilities to create capacity meeting the supply 
of spent fuel to prevent under capacity. 

\subsubsection{Input Variables}
Table \ref{tab:inputs} lists and gives examples of the input 
variables \deploy accepts. 
Essentially, the user must define the facilities controlled by 
\deploy, their respective capacities, the driving commodity, 
its demand equation, deployment driving method, and prediction method 
for supply and demand. 
The user also has the optional option to define supply/capacity buffers 
for each commodity, facility preferences, and facility constraints. 
In-depth descriptions of the deployment driving method, prediction 
methods, and buffers are provided in the subsequent sections. 

\begin{table}[H]
	\resizebox{\textwidth}{!}{%
	\begin{tabular}{|l|l|p{7cm}|}
	\hline
											  & \textbf{Input Parameter}                                                           & \textbf{Examples}                                                                                                          \\ \hline
	\multirow{5}{*}{\textbf{Required}} & Demand driving commodity                                                           & Power, Fuel, Plutonium, etc.                                                                                                                      \\ \cline{2-3} 
											  & Demand equation                                                                    & P(t) = 10000, sin(t), 10000*t                                                                                                                 \\ \cline{2-3} 
											  & Facilities it controls                                                             & Fuel Fab, LWR reactor, SFR reactor, Waste repository, etc.                                                                                                      \\ \cline{2-3} 
											  & Capacities of the facilities                                                       & 3000 kg, 1000 MW, 50000 kg                                                                                                     \\ \cline{2-3} 
											  & Prediction method                                                                  & \begin{tabular}[c]{@{}l@{}}Power: fast fourier transform\\ Fuel: moving average\\ Spent fuel: moving average\end{tabular} \\ \cline{2-3} 
											  & Deployment driven by & Installed Capacity/Supply                                                                                                                    \\ \hline
	\multirow{4}{*}{\textbf{Optional}} & Supply/Capacity Buffer type                                                                        & Absolute                                                                                                                  \\ \cline{2-3} 
											  & Supply/Capacity Buffer size                                                                        & \begin{tabular}[c]{@{}l@{}}Power: 3000 MW\\ Fuel: 0 kg \\ Spent fuel: 0 kg\end{tabular}                                   \\ \cline{2-3} 
											  & Facility preferences                                                               & \begin{tabular}[c]{@{}l@{}}LWR reactor = 100-t\\ SFR reactor = t-100 \end{tabular}          \\ \cline{2-3} 
											  & Facility constraint                                                              & SFR reactor constraint = 5000kg of Pu            \\ \hline	
			
											\end{tabular}%
	}
	\caption{\deploy's required and optional input parameters with examples.}
	\label{tab:inputs}
    \end{table}

    \subsubsection{Deployment Driving Method}
    The user has the choice of deploying facilities based on the difference 
    between predicted supply and demand, or predicted demand and 
    installed capacity. 
    There are two advantages of using installed capacity over predicted 
    supply. 
    First, to prevent over deployment of facilities that have an
    intermittent supply. 
    For example, reactor facilities have a periodic refueling time. 
    A user might not want \deploy to deploy more reactor facilities 
    to make up for the lack of power supply caused by the gap in 
    supply during refueling. 
    Second, to prevent infinite deployment of a facility that uses 
    a commodity that is no longer available in the simulation. 
    For example, in a transition scenario from \gls{LWR}s to \gls{SFR}s, 
    the reprocessing plant that fabricates \gls{SFR} fuel might demand 
    Pu after the inventory accumulated by \gls{LWR}s is used up 
    and there are no more \gls{LWR} facilities to generate Pu. 
    This will result in \deploy deploying infinite reprocessing 
    facilities to generate \gls{SFR} fuel despite the lack of input Pu 
    to generate it. 
    This can be avoided by using \deploy's facility constraint capability 
    to constrain \gls{SFR} deployment until a sizable inventory of Pu 
    is accumulated in the simulation. 
    
    \subsubsection{Supply/Capacity Buffer}
    In \texttt{DemandDrivenDeploymentInst}, the user has the option 
    to provide a supply buffer for each commodity so that 
    \deploy will deploy facilities to meet predicted demand and the
    additional buffer value. 
    In \texttt{SupplyDrivenDeploymentInst}, the user has the option 
    to provide a capacity buffer to specific commodities so that 
    \deploy will deploy facilities to meet predicted supply and the
    additional buffer.
    For example, the user could set the power commodity's supply buffer 
    to be 2000 MW. 
    If predicted demand is 10000 MW, \deploy will deploy reactor 
    facilities to meet the predicted demand and supply buffer, resulting 
    in a power supply of 12000 MW.  
    The buffer can be defined as a percentage (equation \ref{eq:perc}) 
    or absolute value (equation \ref{eq:abs}). 
    
    \begin{equation}
        \label{eq:perc}
        S_{pwb} = S_{p}*(1+d)
    \end{equation}
    \begin{equation}
        \label{eq:abs}
        S_{pwb} = S_{p}+a
    \end{equation}
    where $S_{pwb}$ is predicted supply/capacity with buffer, 
    $S_p$ is the predicted supply/capacity without buffer, 
    $d$ is the percentage value in decimal form, 
    and $a$ is the absolute value of the buffer. 
    
    Using a combination of this buffer capability with the 
    installed capacity deployment driving method in a transition 
    scenario simulation is effective in minimizing undersupply of a 
    commodity without having excessive over supply. 
    This is demonstrated in section \ref{sec:demo}. 
    
    \subsubsection{Preferences}
    % Need to explain the order of preferences for deployment 
    % Constraint, pref, minimize number of facilities and minimize 
    % over supply 
    The user has the option to provide each facility with
    a time dependent preference equation that governs preference for 
    that facility compared to other facilities that provide the same 
    commodity. 
    In the example for facility preferences in table \ref{tab:inputs}, 
    the \gls{LWR} reactor has a preference of $100-t$ and the 
    \gls{SFR} reactor has a preference of $t-100$. 
    Thus, the \gls{LWR} is preferred before time step 100 and \gls{SFR}
    is preferred after. 
    
    The user also has the option to provide each facility with a 
    commodity constraint. 
    In the example for facility constraint in table \ref{tab:inputs}, 
    the \gls{SFR} has a commodity constraint of 5000kg of Pu. 
    This constrains \gls{SFR} deployment by the size of the Pu inventory 
    in the simulation. 
    Once, the 5000kg Pu inventory is first met, \gls{SFR} reactors can 
    henceforth be deployed. 
    
    One of the key issues faced in transition scenarios is the lack 
    of Pu in a scenario that results in idle advanced reactor capacity. 
    Therefore, the facility preferences and constraint capabilities 
    are useful and necessary for modeling transition scenarios. 
    An ideal transition year is selected using the facility 
    preferences, however the transition will only begin when there 
    is sufficient Pu inventory (set by facility constraint) 
    to avoid Pu shortages. 
    
    Therefore, when \deploy predicts an undersupply of a commodity, 
    it deploys available facilities to meet the predicted demand. 
    It will deploy the facility with the highest preference first, 
    unless it does not meet it's constrained criteria, then it will 
    deploy the second most, and so on. 
    If the facilities do not have preferences or constraints, \deploy 
    will deploy the available facilities to minimize the number of 
    deployed facilities while minimizing oversupply of the commodity.


\section{Dymond}