\chapter{Methods}
In this section, the \glspl{NFCSim}, \Cyclus and Dymond, will 
be described and evaluated on their capability of setting up 
successful transition scenarios. 

\section{\Cyclus}
\Cyclus is an agent-based nuclear fuel cycle simulation framework 
\cite{huff_fundamental_2016}. 
In \Cyclus, each entity (i.e. Region, Institution, or Facility) in the 
fuel cycle is an agent. 
Region agents represent geographical or political areas that institution
and facility agents can be grouped into. 
Institution agents control the 
deployment and decommission of facility agents 
and represents legal operating organizations such as a 
utility, government, etc. \cite{huff_fundamental_2016}. 
Facility agents represent nuclear fuel cycle facilities. 
\Cycamore \cite{carlsen_cycamore_2014}
provides facility agents to represent process physics of various 
components in the nuclear fuel cycle (e.g. mine, fuel enrichment 
facility, reactor). 

In 2016, there was a push to understand and evaluate the 
transition from the initial EG01 state to promising future 
end-states \cite{feng_standardized_2016}.
Previously in \Cyclus, reactor facilities are automatically 
deployed to meet a user-defined power demand. 
However, it is up to the user to define a deployment scheme of 
supporting facilities to ensure that there is no gap in the supply 
chain that results in idle reactor capacity. 
Therefore, previously in \Cyclus, some users choose to set 
support facilities to have an infinite capacity to avoid this 
issue, but this is an inaccurate representation of reality. 
It is straightforward to manually determine a deployment scheme 
for a once-through fuel cycle, however, it is difficult to effectively 
implement for complex closed fuel cycle scenarios.  
To successfully conduct analysis of the time-dependent transition
analyses, it is necessary to develop \gls{NFCSim} tools to  
automate setting up of transition scenarios. 
Therefore, Demand-Driven Cycamore Archetypes project
(NEUP-FY16-10512) was initiated to develop demand-driven deployment 
capabilities in \Cyclus.

This capability is added as a \Cyclus Institution
agent that deploys facilities to meet the front-end and back-end 
fuel cycle demands based on a user-defined commodity demand. 
This demand-driven deployment capability is called 
\deploy. 

\subsection{\deploy}





\section{Dymond}