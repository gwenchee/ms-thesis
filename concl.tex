\chapter{Conclusion}
% gap, methods, accomplishments 
\section{Contributions}
The present nuclear fuel cycle in the United States is a once-through 
fuel cycle of LWRs with no used fuel reprocessing. 
This nuclear fuel cycle faces cost, safety, proliferation, and spent 
nuclear fuel challenges that hinder large-scale nuclear power deployment. 
The U.S Department of Energy identified future 
nuclear fuel cycles, involving continuous recycling of co-extracted U/Pu or 
U/TRU in fast and thermal spectrum reactors, that may overcome these challenges.
These transition scenarios have been modeled previously in the following 
nuclear fuel cycle simulators \cite{feng_standardized_2016,bae_standardized_2019}: 
ORION, DYMOND, VISION, MARKAL, and \Cyclus. 
However, for many nuclear fuel cycle simulators, the user is required to 
define a deployment scheme for all supporting facilities to avoid any 
supply chain gaps or resulting idle reactor capacity. 
Manually determining a deployment scheme for a once-through 
fuel cycle is straightforward; however, for complex fuel cycle 
scenarios, it is not. 
This thesis developed the capability, \deploy, in \Cyclus that automatically deploys 
fuel cycle facilities to meet user-defined power demand. 
This thesis demonstrated \deploy's set up of the transition from the current 
LWR fleet to a closed fuel cycle in which transuranics are recycled to fuel 
MOX LWRs and SFRs.  
When \deploy used the FFT prediction method and a 2000MW power buffer size, 
it successfully deployed reactors and supporting fuel cycle facilities to 
meet a linearly increasing power demand with minimal power undersupply. 

Historically, transition scenarios were modeled 
using a single model in which the transition scenario future 
is modeled based on assumptions made about various input parameters 
such as facility size, length of used fuel cooling time, etc. 
In reality, the transition process inevitably diverges from the 
modeled scenario. 
This thesis coupled nuclear fuel cycle simulators, \Cyclus and 
DYMOND, with Dakota, a sensitivity analysis tool.
This work demonstrated one-at-a-time, synergistic, and 
global sensitivity analysis with \Cyclus-Dakota and DYMOND-Dakota,
to understand the interdependence of input parameters on 
transition scenario performance. 
The transition year input parameter was the most influential
to the final depleted uranium and total idle reactor capacity 
performance metrics. 
The fleet share ratio and cooling time input parameters 
were the most influential to the final high level waste amount in the 
simulation. 

This thesis compared \Cyclus-Dakota's and DYMOND-Dakota's sensitivity analysis 
capabilities 
and concluded that automated deployment of supporting fuel cycle 
facilities is crucial for conducting transition scenario sensitivity analyses, 
to ensure that the simulation 
adapts to the new parameters by minimizing idle reactor capacity.
This work demonstrated that the most influential input variables 
on each performance 
metric could be determined through global sensitivity 
analysis, narrowing-down the one-at-a-time and 
synergistic sensitivity analyses that need to be conducted.  
If not, the analyses might include more than 20 input parameters' 
one-at-a-time and synergistic sensitivity analyses, 
resulting in difficulties concisely merging them to optimize your 
transition scenario parameters.

\section{Suggested Future Work}
Using the tools developed and demonstrated in this thesis, a comprehensive 
sensitivity analysis of the EG01-EG30 transition scenario or any scenario of interest 
can be conducted. 
The comprehensive sensitivity analysis should begin with a global
sensitivity analysis to determine the influential input parameters 
on the output parameters of interest. 
Following that, one-at-a-time and synergistic sensitivity analysis can be used to do 
further analysis of the influential and interdependent input parameters, respectively. 
Thus, the trends and quantitative impacts of influential input variables on the performance 
metrics can be determined.  