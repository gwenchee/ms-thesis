\chapter{Conclusion}
The present nuclear fuel cycle in the United States is a once-through 
fuel cycle of light water reactors with no used fuel reprocessing. 
This nuclear fuel cycle faces cost, safety, proliferation, and spent nuclear fuel
challenges that hinder large-scale nuclear power deployment. 
The U.S Department of Energy identified future 
nuclear fuel cycles that may overcome these challenges.  
One of the most promising nuclear fuel cycles is evaluation group (EG) 30, 
which reprocesses transuranics to fuel \gls{MOX} \glspl{PWR}
and \glspl{SFR}. 
Modeling the transition from our current fleet to EG30 was historically 
conducted using a single model in which the transition scenario future 
is predicted based on assumptions made about various input parameters 
such as facility size, length of used fuel cooling time, etc. 
In reality, the transition process inevitably diverges from the 
modeled scenario. 
To better understand the subtleties of the EG01-EG30 transition,
in this work, we conducted sensitivity analysis studies to 
understand the impact of changes in input parameters on the transition 
and its final state. 

We learned that comprehensive sensitivity analysis studies of nuclear fuel cycles 
are only feasible if the nuclear fuel cycle simulator can 
automatically deploy supporting fuel cycle facilities 
to ensure minimal idle reactor capacity. 
This automatic deployment capability makes nuclear fuel cycle simulators 
flexible and resilient to 
perturbations in transition scenario input parameters. 
In this work, we developed the \deploy capability in \Cyclus to 
automatically deploy fuel cycle facilities to meet user-defined 
power demand. 
We demonstrated the use of \deploy to set up the EG01-EG30 transition scenario,
and found that the \texttt{FFT} prediction method was most effective at 
minimizing power undersupply. 

We used \deploy for the sensitivity analysis studies, and it has 
proven to be flexible in minimizing idle reactor capacity for 
transition scenarios with varying input parameters. 
DYMOND requires the user to manually set up a correct fuel management 
strategy to ensure a successful transition scenario with minimal 
idle reactor capacity. 
Therefore, when DYMOND is used for sensitivity analysis studies, it tends 
to produce transition scenario simulations with idle reactor capacity 
because the fuel management strategy was optimized manually for specific 
scenario parameters and is not robust to changes in the input parameters.

Sensitivity analyses for large systems such as nuclear 
fuel cycle transition scenarios are challenging to conduct 
since they involve a vast input parameter space with
upwards of 20 input variables. 
Previous work toward nuclear fuel cycle transition scenario sensitivity analysis 
includes a one-at-a-time sensitivity analysis of each input parameter 
to see the impact on the output variables of interest \cite{noauthor_effects_2017}.
However, this type of sensitivity analysis fails to capture the 
synergistic effects of the input parameters on the system.  
In this work, we coupled nuclear fuel cycle simulators, \Cyclus and 
DYMOND, with Dakota, a sensitivity analysis tool. 
By coupling the nuclear fuel cycle simulators with Dakota, 
many types of sensitivity analyses, 
can be conducted together to understand the subtle
interdependence of input parameters on transition scenario
performance. 

In this work, we demonstrated the use of \Cyclus-Dakota coupling 
and DYMOND-Dakota coupling to conduct one-at-a-time, synergistic, and 
global sensitivity analysis. 
We found that the transition year input parameter was the most influential
to the final depleted uranium and total idle reactor capacity performance 
metrics, however, not it was not influential at all to the amount of final high 
level waste in the simulation. 
We found that fleet share ratio and cooling time input parameters 
were the most influential to the amount of final high 
level waste in the simulation. 
This work demonstrates that it is crucial to initially determine evaluation metrics and 
streamline the input variables of interest through global sensitivity 
analysis; if not, the resulting analysis might include one-at-a-time and 
synergistic sensitivity analysis of more than 20 input parameters, 
and difficulties merging them to optimize your transition scenario 
effectively.

\section{Future Work}
Using the tools developed and demonstrated in this thesis, a comprehensive 
sensitivity analysis of the EG01-EG30 transition scenario can be conducted. 
The comprehensive sensitivity analysis should begin with a global
sensitivity analysis to determine the influential input parameters 
on the output parameters of interest. 
Following that, one-at-a-time and synergistic sensitivity analysis can be used to do 
further analysis of the influential and interdependent input parameters, respectively. 
Thus, we can determine the trends and quantitative impacts of influential input 
variables on the performance metrics.  
