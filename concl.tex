\chapter{Conclusion}

% Research Problem and how they were addressed. 
The present nuclear fuel cycle in the United States is a once-through 
fuel cycle of light water reactors in which there is no reprocessing 
of the used fuel. 
This nuclear fuel cycle faces cost, safety, proliferation, and waste 
challenges that hinder large scale nuclear power deployment. 
The U.S Department of Energy chartered a study and identified future 
nuclear fuel cycles that overcome these challenges.  
One of the most promising nuclear fuel cycles is evaluation group (EG) 30, 
which reprocesses used fuel's transuranics to fuel \gls{MOX} \glspl{PWR}
and \glspl{SFR}. 
Modeling the transition from our current fleet to EG30 was historically 
conducted using a single model in which the transition scenario future 
is predicted based on assumptions made about various input parameters 
such as facility size, length of used fuel cooling time, etc. 
In reality, the real transition process inevitably diverges from the 
modeled scenario. 
To better understand the subtleties of the EG01-EG30 transition,
in this work, we conduct sensitivity analysis studies to 
understand the impact of changes in input parameters on the transition 
and its final state. 

We learned that sensitivity analysis studies of nuclear fuel cycles 
are only effective if the nuclear fuel cycle simulator can 
automatically deploy supporting fuel cycle facilities 
to ensure that there is minimal idle reactor capacity. 
NFC simulators must be flexible and resilient to 
perturbations in transition scenario input parameters. 
In this work, we developed the \deploy capability in \Cyclus to 
automatically deploy fuel cycle facilities to meet user-defined 
power demand. 
We used \deploy for the sensitivity analysis studies, and \Cyclus 
is shown to be flexible in setting up successful transition 
scenarios, in which idle reactor capacity is minimized
when input parameters are slightly varied.
DYMOND can successfully set up a transition scenario,= if the 
user ensures to set up a fuel management strategy correctly. 
However, for sensitivity analysis studies, it is less effective, 
because the fuel management strategy was optimized manually for specific 
scenario parameters, resulting in idle reactor capacity when 
an input parameter changes during sensitivity analysis.

Sensitivity analysis studies for large systems such as nuclear 
fuel cycle transition scenarios are challenging to conduct 
since there is a vast input sample space with an
upwards of 20 input parameters that could be interdependent. 
Previous work towards sensitivity analysis of the nuclear fuel cycle 
conducted a one-at-a-time sensitivity analysis of each input parameter 
to see the impact on the output variables of interest.
However, this type of sensitivity analysis fails to capture the 
synergistic effects of the input parameters on the system.  
In this work, we coupled nuclear fuel cycle simulators, \Cyclus and 
DYMOND with Dakota, a sensitivity analysis tool. 
By doing this, we give these nuclear fuel cycle simulator codes 
a new dimension to performing sensitivity analysis.
With the use of one tool, many types of sensitivity analyses, 
such as one-at-a-time, synergistic, and global 
can be conducted together to understand the subtlety of the 
interdependence of input parameters on a transition scenario's 
transition period and final state. 
In this work, we demonstrated the use of \Cyclus-Dakota coupling 
and DYMOND-Dakota coupling to conduct one-at-a-time, synergistic, and 
global sensitivity analysis. 
Sensitivity analysis studies cover a vast input space. 
It is crucial to initially determine your evaluation metrics and 
streamline the input variables of interest through global sensitivity 
analysis; if not, the resulting analysis might include one-at-a-time and 
synergistic sensitivity analysis of more than 20 input parameters, 
and difficulties merging them to optimize your transition scenario 
effectively.

\section{Future Work}
Using the tools developed and demonstrated in this thesis, a comprehensive 
sensitivity analysis of the EG01-EG30 transition scenario could be conducted. 
The comprehensive sensitivity analysis should begin with a global
sensitivity analysis to determine the influential input parameters 
on the output parameters of interest. 
Following that, OAT and synergistic sensitivity analysis tools can 
be used to do further analysis for the influential input parameters and 
input parameters that are shown to have interdependence to 
determine the trends and quantitative impacts of specific input variables 
on output variables.  

