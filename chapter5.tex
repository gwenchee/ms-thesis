\chapter{Results: Sensitivity Analysis}
This chapter consists of the sensitivity analysis results of the 
EG01-30 \gls{NFC} transition scenario. 
Various types of \gls{SA} are conducted
using the Dakota-\Cyclus and Dakota-Dymond coupling. 
This chapter will be broken into five sections: 
\begin{enumerate}
    \item Transition Scenario Specification 
    \item Sensitivity Analysis Evaluation Metrics 
    \item One-at-a-time \gls{SA} results 
    \item Synergistic \gls{SA} results
    \item Global \gls{SA} results 
\end{enumerate}

\section{Sensitivity Analysis: Transition Scenario Specification}
% improve this section 
The same EG01-30 transition scenario is used for the Dymond 
sensitivity analysis and \Cyclus sensitivity analysis. 
The differences lie in the nuances of certain input variables but 
it is mostly the same. 

\subsection{Dymond}
Dymond transition scenario sensitivity analysis is conducted 
using the specifications of the Dymond OECD benchmark transition 
scenario presented at the 17th Meeting of Expert Group on Advanced 
Fuel Cycle Scenarios in France in 2017 
\cite{oecd_nuclear_energy_agency_wpfc_nodate}. 
The transition scenario is from the current \gls{PWR} fleet to 
a mixed fleet of \gls{MOX} \glspl{PWR} and \glspl{SFR}. 


\begin{table}[]
    \caption{OECD Benchmark Transition Scenario
	Specifications \cite{oecd_nuclear_energy_agency_wpfc_nodate}}
	\label{tab:dymondinputs}
    \footnotesize
    \begin{tabularx}{\textwidth}{LL}
    \hline
                               \textbf{Input Parameter}            & \textbf{Value}            \\ \hline
    Demand driving commodity   & Power              \\
                               Demand equation {[}TWhe/y{]}   & 430        \\
                               Transition Start Date [Year] & 80\\ 
                               Fleet share ratio [\%] & \gls{MOX} \gls{PWR}: 15\%, \gls{SFR}: 85\%\\ \hline
    \end{tabularx}%
    \end{table}

\subsection{\Cyclus}
\Cyclus transition scenario sensitivity analysis will be conducted 
on the EG01-30 transition 
scenario for linearly increasing power demand (described in 
the section \ref{sec:eg01-30}). 
Figure \ref{fig:30flow} shows the facility and mass flow 
for this transition scenario in \Cyclus. 
Tables \ref{tab:bestinputs} and \ref{tab:facinputs}
shows the input parameters for \deploy and 
in the transition scenario. 
The \texttt{reactor} facility used in the \Cyclus simulation 
is a recipe reactor; it accepts a fresh fuel recipe and outputs 
a spent fuel recipe. 
The recipes used for the \gls{LWR}, \gls{MOX} \gls{LWR}, and 
\gls{SFR} are based on recipes generated by VISION 
\cite{chee_arfc/transition-scenarios_2018}
that closely match EG30 scenario specifications in 
Appendix B of the \gls{DOE} Evaluation and Screening Study 
(E\&S study) \cite{wigeland_nuclear_2014}. 
% describe input parameters and what was changed. 

\begin{table}[]
    \caption{\Cyclus facility input parameters for
	EG01-EG30 transition scenario
	that minimizes undersupply for power and minimizes 
	the undersupply and under capacity for other facilities. }
	\label{tab:facinputs}
    \footnotesize
    \begin{tabularx}{\textwidth}{L|LL}
        \hline
        \textbf{Facility}                 & \textbf{Input Parameter}                    & \textbf{Value} \\ \hline
        \multirow{4}{*}{\textbf{LWR}}     & Lifetime {[}months{]}              & 960   \\
                                 & Cycle time {[}months{]}            & 18    \\
                                 & Refuel time {[}months{]}           & 1     \\
                                 & Rated Power {[}MWe{]}              & 1000  \\ \hline
        \multirow{2}{*}{\textbf{MOX LWR}} & Lifetime {[}months{]}              & 960   \\
                                 & Cycle time {[}months{]}            & 18    \\
                                 & Refuel time {[}months{]}           & 1     \\
                                 & Rated Power {[}MWe{]}              & 1000  \\ \hline
        \multirow{4}{*}{\textbf{SFR}}     & Lifetime {[}months{]}              & 720   \\
                                 & Cycle time {[}months{]}            & 12    \\
                                 & Refuel time {[}months{]}           & 1     \\
                                 & Rated Power {[}MWe{]}              & 333   \\ \hline
        \textbf{Cooling Pools}            & Used fuel storage time {[}years{]} & 3  \\ \hline
        \end{tabularx}
    \end{table}

\section{Sensitivity Analysis: Evaluation Metrics}
To determine the basis of comparison for sensitivity analysis 
of a \gls{NFC} transition scenario, important optimization 
metrics must be defined and associated with output variables.

In the E\&S study 
\cite{wigeland_nuclear_2014}, transition 
scenarios were evaluated for nine metrics: nuclear waste 
management, proliferation risk, nuclear material security risk, 
safety, environmental impact, resource utilization, development 
and deployment risk, institutional issues, financial risk and 
economics. 
These nine metrics can be further grouped and narrowed down to 
four categories \cite{passerini_systematic_2014}: environmental 
impact, economics, proliferation risk and resource utilization. 

To quantify the impact of the variation of an input parameter 
on an output parameter, it is necessary to define output indicators 
to measure this impact \cite{noauthor_effects_2017}. 
Output indicators are introduced because the output variables
are a time series resulting in a need for a single value that 
is representative of the output parameter's time series.  
Five types of output indicators are introduced 
\cite{noauthor_effects_2017}: 
(1) final value at end of simulation
(2) maximum value during simulation, 
(3) minimum value during simulation, 
(4) cumulative sum over the whole simulation, and 
(5) slope of the final 50 points of the simulation.
Depending on the nature of the output parameter, a different 
output indicator will be used. 
The slope indicator determines if a variable is increasing or 
decreasing at the end of the simulation. 
Indicators with transition refer to the indicator value during the 
transition period. 
The transition period is defined as from the year when the first 
transition reactor is built till the year of last idle capacity. 

Table \ref{tab:category-output-DD} shows the selected 
output variables used in this work and their associated 
evaluation metrics. 

% update for cyclus and dymond when decided
\begin{table}[]
    \centering
    \caption {Evaluation metrics and their associated output 
    variables for sensitivity analysis.}
	\label{tab:category-output-DD}
        \footnotesize
        \begin{tabularx}{\textwidth}{l|LL}	
            	\hline
            \textbf{Evaluation Metrics} & \textbf{Output Variable} & \textbf{Indicators}\\
            \hline
            \textbf{Waste Management} & \begin{tabular}[c]{@{}l@{}}Total \gls{HLW} Inventory\\ Depleted Uranium\end{tabular} & \begin{tabular}[c]{@{}l@{}}Final \& Transition Final\\ Final \& Transition Final\end{tabular}\\
            \hline
            \textbf{Proliferation Risk} &  \begin{tabular}[c]{@{}l@{}}Pu in cooling pools\\ Separated Pu in storage \\ Separated Pu in HLW \\Fissile Pu in cooling pools \\ 
            Fissile Separated Pu in storage \\ Fissile Separated Pu in HLW \\\end{tabular} & 
            \begin{tabular}[c]{@{}l@{}} Max, Year, Quality, Slope\\ Max, Year, Quality, Slope \\ Max, Year, Quality, Slope \\ Max, Year, Slope \\ Max, Year, Slope \\ Max, Year, Slope\end{tabular} \\
            
            \hline
            \textbf{Resource Utilization} & Uranium ore consumed & Sum, Transition Sum\\
            \hline
            \textbf{Goodness of Transition} & \begin{tabular}[c]{@{}l@{}}Total Idle Capacity\\ Date of Idle Capacity \\ Length of transition\end{tabular} & \begin{tabular}[c]{@{}l@{}}Sum \\ Final \\ -\end{tabular} \\ 
            \hline
            \end{tabularx}
\end{table}

The operational conditions for the advanced reactors and
the specifics of the transition scenario are not 
fixed since the fuel cycle simulator is modeling future 
trajectories. 
The input parameters that are varied in the
transition scenarios are: 
\begin{itemize}
    \item Length of cooling time for discharged fuel 
    \item Fleet share ratio of PWR MOX and SFR reactors 
	\item Introduction date of advanced reactor technology
\end{itemize}

The sensitivity analysis was conducted for both Dymond and \Cyclus. 

\section{One-at-a-time Sensitivity Analysis}
Used fuel cooling time for all cooling pools was varied from 
0 to 7 years for the OECD benchmark transition scenario. 
Each of these simulations are compared based on the evaluation 
metrics described in table \ref{tab:category-output-DD}.

\subsection{Length of cooling time for discharged fuel}
\subsubsection{\textbf{Dymond}}
Table \ref{tab:DD-SA-results} shows the absolute values of 
output variables associated with the environmental impact, 
resource utilization, and goodness of transition evaluation 
metrics for each scenario. 
Table \ref{tab:DD-SA-perc} shows each scenario's percentage 
difference compared with the base case of `Cooling Time = 2 years'
scenario.

\begin{table}[]
    \centering
    \caption{Dymond: Assessment of how variation of used fuel cooling times
    impacts evaluation metrics for OECD benchmark
	transition scenario with different used fuel cooling times.}
	\label{tab:DD-SA-results}
        \footnotesize
        \begin{tabularx}{\textwidth}{L|LL|L|LLL}	
            \hline
            \textbf{} & \multicolumn{2}{c|}{\textbf{Environmental Impact}}                                                                                                                                                                                                                                                      & \textbf{Resource Utilization}                                                                                        & \multicolumn{3}{c}{\textbf{Goodness of Transition}}                                                                                                                                                                                 \\ \hline
\textbf{Scenario: Used Fuel Cooling Time} & \textbf{Final HLW [kg] } & \textbf{Final Depleted U [kg]} &  \textbf{Total U Ore [kg]}  & \textbf{Total Idle Capacity[kg]} & \textbf{Year of Final Idle Capacity} & \textbf{Length of Transition [years]} \\ \hline
CT = 0  &           1103.2 &                             916933.4 &                       16188.8 &                                    30148.8 &                      301 &                     227 \\ 
CT = 1  &           1101.6 &                             916618.2 &                       16188.8 &                                    30148.8 &                      301 &                     227 \\ 
CT = 2  &           1105.7 &                             916237.60 &                       16188.8 &                                    30148.8 &                      301 &                     227 \\ 
CT = 3  &           1108.3 &                             916268.7 &                       16188.8 &                                   256588.8 &                      301 &                     227 \\ 
CT = 4  &           1099.8 &                             916962.4 &                       16188.8 &                                 1338604.8 &                      301 &                     227 \\ \hline
\end{tabularx}%
\end{table}

\begin{table}[]
    \caption{Dymond: Sensitivity Analysis Results for OECD benchmark
    transition scenario with different used fuel cooling times.
    The numbers in the table represent by how many \% an output variable 
    from each scenario differs from the base case.}
    \label{tab:DD-SA-perc}
    \footnotesize
    \begin{tabularx}{\textwidth}{L|LL|L|LLL}	
		\hline
        \textbf{} & \multicolumn{2}{c|}{\textbf{Environmental Impact}}                                    & \textbf{Resource Utilization}                                                                                       & \multicolumn{3}{c}{\textbf{Goodness of Transition}}                                                                                                                                                                                 \\ \hline
        \textbf{Scenario: Used Fuel Cooling Time} & \textbf{Final HLW [kg] } & \textbf{Final Depleted U [kg]} &  \textbf{Total U Ore [kg]}  & \textbf{Total Idle Capacity[kg]} & \textbf{Year of Final Idle Capacity} & \textbf{Length of Transition [years]} \\ \hline
        CT = 0  &             \cellcolor[HTML]{67FD9A}-0.221736 &                                   \cellcolor[HTML]{67FD9A}0.075939 &                                                            \cellcolor[HTML]{67FD9A}0.0 &                 \cellcolor[HTML]{67FD9A}0.000000 &                                           \cellcolor[HTML]{67FD9A}0.0 & \cellcolor[HTML]{67FD9A}0.0 \\
		CT = 1  &             \cellcolor[HTML]{67FD9A}-0.363540 &                                    \cellcolor[HTML]{67FD9A}0.041532 &                                                           \cellcolor[HTML]{67FD9A}0.0 &                 \cellcolor[HTML]{67FD9A}0.000000 &                                          \cellcolor[HTML]{67FD9A}0.0 & \cellcolor[HTML]{67FD9A}0.0 \\ 
		CT = 2  &              \cellcolor[HTML]{000000}0.000000 &                                     \cellcolor[HTML]{000000}0.000000 &                                                              \cellcolor[HTML]{000000}0.0 &                 \cellcolor[HTML]{000000}0.000000 &                                         \cellcolor[HTML]{000000}0.0 & \cellcolor[HTML]{000000}0.0 \\ 
		CT = 3  &              \cellcolor[HTML]{67FD9A}0.234524 &                                    \cellcolor[HTML]{67FD9A}0.003389 &                                                              \cellcolor[HTML]{67FD9A}0.0 &               \cellcolor[HTML]{FD6864}751.074670 &                                         \cellcolor[HTML]{67FD9A}0.0 & \cellcolor[HTML]{67FD9A}0.0 \\ 
		CT = 4  &             \cellcolor[HTML]{67FD9A}-0.529033 &                                   \cellcolor[HTML]{67FD9A}0.079102 &                                                        \cellcolor[HTML]{67FD9A}0.0 &              \cellcolor[HTML]{FD6864}4339.993632 &                                        \cellcolor[HTML]{67FD9A}0.0 & \cellcolor[HTML]{67FD9A}0.0 \\ \hline
	\end{tabularx}%
    
    \resizebox{0.3\textwidth}{!}{
    \fbox{\begin{tabular}{ll}
        \textcolor{green}{$\blacksquare$} & $sensitivity \leq 1\%$ \\
        \textcolor{orange}{$\blacksquare$} & $1\% < sensitivity < 10\%$ \\
        \textcolor{red}{$\blacksquare$} & $sensitivity \geq 10\%$
        \end{tabular}}}
    \end{table}

    
\subsubsection{\textbf{\Cyclus}}
% change to 0,2,4,6,8
\begin{table}[]
    \centering
    \caption{\Cyclus: Assessment of how variation of used fuel cooling times
    impacts evaluation metrics for EG01-30 
	transition scenario.}
	\label{tab:DD-SA-results}
        \footnotesize
        \begin{tabularx}{\textwidth}{L|LL|L|LLL}
            \hline	
            \textbf{} & \multicolumn{2}{c|}{\textbf{Environmental Impact}}                                                                                                                                                                                                                                                      & \textbf{Resource Utilization}                                                                                        & \multicolumn{3}{c}{\textbf{Goodness of Transition}}                                                                                                                                                                                 \\ \hline
\textbf{Scenario: Used Fuel Cooling Time} & \textbf{Final HLW [kg] } & \textbf{Final Depleted U [kg]} &  \textbf{Total U Ore [kg]}  & \textbf{Total Idle Capacity[kg]} & \textbf{Year of Final Idle Capacity} & \textbf{Length of Transition [years]} \\ \hline

               0  & 13223828.1 & 798818620.4      & 143700000000      & 135.1               & 962                     & 2                      \\
                1  & 13189931.3 & 798818620.4      & 143700000000      & 135.1               & 962                     & 2                      \\
                2  & 13073261.2 & 798818620.4      & 143700000000      & 135.1               & 962                     & 2                      \\
                3  & 12959554.5 & 798818620.4      & 143700000000      & 135.1               & 962                     & 2                      \\
                4  & 12906058.3 & 798818620.4      & 143700000000      & 135.1               & 962                     & 2                      \\
                5  & 12927175.5 & 798818620.4      & 143700000000      & 135.1               & 962                     & 2                      \\
                6  & 12795682.5 & 798818620.4      & 143700000000      & 135.1               & 962                     & 2                      \\
                7  & 12806768.1 & 798818620.4      & 143700000000      & 135.1               & 962                     & 2                     \\ \hline
        \end{tabularx}
\end{table}

% add color 
\begin{table}[]
    \caption{\Cyclus: Sensitivity Analysis Results for OECD benchmark
    transition scenario with different used fuel cooling times.
    The numbers in the table represent by how many \% an output variable 
    from each scenario differs from the base case.}
    \label{tab:DD-SA-perc}
    \footnotesize
    \begin{tabularx}{\textwidth}{L|LL|L|LLL}	
		\hline
        \textbf{} & \multicolumn{2}{c|}{\textbf{Environmental Impact}}                                    & \textbf{Resource Utilization}                                                                                       & \multicolumn{3}{c}{\textbf{Goodness of Transition}}                                                                                                                                                                                 \\ \hline
        \textbf{Scenario: Used Fuel Cooling Time} & \textbf{Final HLW [kg] } & \textbf{Final Depleted U [kg]} &  \textbf{Total U Ore [kg]}  & \textbf{Total Idle Capacity[kg]} & \textbf{Year of Final Idle Capacity} & \textbf{Length of Transition [years]} \\ \hline
        0  & 2.29      & 0.0              & 0.0               & 0.0                 & 0.0                     & 0.0                    \\
        1  & 2.03      & 0.0              & 0.0               & 0.0                 & 0.0                     & 0.0                    \\
        2  & 1.13      & 0.0              & 0.0               & 0.0                 & 0.0                     & 0.0                    \\
        3  & 0.25      & 0.0              & 0.0               & 0.0                 & 0.0                     & 0.0                    \\
        4  & -0.16     & 0.0              & 0.0               & 0.0                 & 0.0                     & 0.0                    \\
        5  & 0.0       & 0.0              & 0.0               & 0.0                 & 0.0                     & 0.0                    \\
        6  & -1.02     & 0.0              & 0.0               & 0.0                 & 0.0                     & 0.0                    \\
        7  & -0.93     & 0.0              & 0.0               & 0.0                 & 0.0                     & 0.0                    \\ \hline
        \end{tabularx}%
    
    \resizebox{0.3\textwidth}{!}{
    \fbox{\begin{tabular}{ll}
        \textcolor{green}{$\blacksquare$} & $sensitivity \leq 1\%$ \\
        \textcolor{orange}{$\blacksquare$} & $1\% < sensitivity < 10\%$ \\
        \textcolor{red}{$\blacksquare$} & $sensitivity \geq 10\%$
        \end{tabular}}}
    \end{table}

\subsection{Fleet share ratio of PWR MOX and SFR reactors}
\subsubsection{\textbf{\Cyclus}}
\begin{table}[]
    \centering
    \caption{\Cyclus: Assessment of how variation of fleet share ratio
    of PWR MOX and SFR reactors
    impacts evaluation metrics for EG01-30 transition scenario.}
	\label{tab:DD-SA-results}
        \footnotesize
        \begin{tabularx}{\textwidth}{L|LL|L|LLL}
            \hline	
            \textbf{} & \multicolumn{2}{c|}{\textbf{Environmental Impact}}                                                                                                                                                                                                                                                      & \textbf{Resource Utilization}                                                                                        & \multicolumn{3}{c}{\textbf{Goodness of Transition}}                                                                                                                                                                                 \\ \hline
\textbf{Scenario: PWR MOX Fleet Share [\%]} & \textbf{Final HLW [kg] } & \textbf{Final Depleted U [kg]} &  \textbf{Total U Ore [kg]}  & \textbf{Total Idle Capacity[kg]} & \textbf{Year of Final Idle Capacity} & \textbf{Length of Transition [years]} \\ \hline

0  & 13153061.6 & 798818620.4      & 143700000000    & 135.1               & 962                     & 2                      \\
5  & 13056988.7 & 798818620.4      & 143700000000    & 135.1               & 962                     & 2                      \\
10 & 13051896.3 & 798818620.4      & 143700000000    & 135.1               & 962                     & 2                      \\
15 & 12959554.1 & 798818620.4      & 143700000000    & 135.1               & 962                     & 2                      \\
20 & 13002120.9 & 798818620.4      & 143700000000    & 135.1               & 962                     & 2                     \\ \hline
        \end{tabularx}
\end{table}


\subsection{Introduction date of advanced reactor technology}

\section{Synergistic Sensitivity Analysis}

\section{Global Sensitivity Analysis}