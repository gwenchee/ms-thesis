\chapter{Results}
This chapter will give results on the sensitivity of various 
input parameters of a \gls{NFC} transition scenario. 
Various types of \gls{SA} are conducted. 
The sensitivity analysis was conducted using the Dakota-\Cyclus
and Dakota-Dymond coupling. 
This chapter is broken into five main sections: 
\begin{enumerate}
    \item Sensitivity Analysis Evaluation Criteria 
    \item Transition Scenario Specification 
    \item One-at-a-time \gls{SA} results 
    \item Synergistic \gls{SA} results
    \item Global \gls{SA} results 
\end{enumerate}


\section{Sensitivity Analysis Evaluation Criteria}
Table \ref{tab:category-output-DD} shows which output variables 
the evaluation criteria are associated with. 

\begin{table}[]
    \centering
	\caption {Evaluation criteria and their associated output variables for Dymond-Dakota sensitivity analysis.}
	\label{tab:category-output-DD}
        \footnotesize
        \begin{tabularx}{\textwidth}{l|LL}	
            	\hline
            \textbf{Evaluation Metrics} & \textbf{Output Variable} & \textbf{Indicators}\\
            \hline
            \textbf{Waste Management} & \begin{tabular}[c]{@{}l@{}}Total \gls{HLW} Inventory\\ Depleted Uranium\end{tabular} & \begin{tabular}[c]{@{}l@{}}Final \& Transition Final\\ Final \& Transition Final\end{tabular}\\
            \hline
            \textbf{Proliferation Risk} &  \begin{tabular}[c]{@{}l@{}}Pu in cooling pools\\ Separated Pu in storage \\ Separated Pu in HLW \\Fissile Pu in cooling pools \\ 
            Fissile Separated Pu in storage \\ Fissile Separated Pu in HLW \\\end{tabular} & 
            \begin{tabular}[c]{@{}l@{}} Max, Year, Quality, Slope\\ Max, Year, Quality, Slope \\ Max, Year, Quality, Slope \\ Max, Year, Slope \\ Max, Year, Slope \\ Max, Year, Slope\end{tabular} \\
            
            \hline
            \textbf{Resource Utilization} & Uranium ore consumed & Sum, Transition Sum\\
            \hline
            \textbf{Goodness of Transition} & \begin{tabular}[c]{@{}l@{}}Total Idle Capacity\\ Date of Idle Capacity \\ Length of transition\end{tabular} & \begin{tabular}[c]{@{}l@{}}Sum \\ Final \\ -\end{tabular} \\ 
            \hline
            \end{tabularx}
\end{table}

To quantify the impact of the variation of an input parameter 
on an output parameter, it is necessary to define output indicators 
to measure this impact \cite{noauthor_effects_2017}. 
Output indicators are introduced because the output variables
are a time series resulting in a need for a single value that 
is representative of the output parameter's time series.  
Five types of output indicators are introduced 
\cite{noauthor_effects_2017}: 
(1) final value at end of simulation
(2) maximum value during simulation, 
(3) minimum value during simulation, 
(4) cumulative sum over the whole simulation, and 
(5) slope of the final 50 points of the simulation.
Depending on the nature of the output parameter, a different 
output indicator will be used. 
The slope indicator determines if a variable is increasing or 
decreasing at the end of the simulation. 
Indicators with transition refer to the indicator value during the 
transition period. 
The transition period is defined as from the year when the first 
transition reactor is built till the year of last idle capacity. 

\subsection{Transition Scenario Specification}

\subsection{One-at-a-time Sensitivity Analysis}

\subsection{Synergistic Sensitivity Analysis}

\subsection{Global Sensitivity Analysis}