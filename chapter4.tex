\chapter{Successful Transition Scenarios}

\section{Demonstration}
This section will demonstrate and evaluate 
setting up of successful transition scenarios 
with \Cyclus and DYMOND. 
In this work, the overall measure of a \gls{NFCSim}'s 
capability to set up successful transition scenarios 
is it's ability to set up transition scenarios using 
input parameters that are slightly perturbed, resulting
in all simulations having no undersupply of power. 
\Cyclus and DYMOND will be evaluated based on this 
overall measure. 

\subsection{\Cyclus}

\subsection{DYMOND}

% difference between cyclus and dymond 
% cyclus is more flexible but harder to use 
% dymond is rigid in the type of scenarios to run 
% more comparisons of cyclus and dymond in this section 
%or after demonstration. 

\section{Major Takeaways}
In this section, \Cyclus and DYMOND's capabilities of setting up 
successful transition scenarios were evaluated. 
\Cyclus is shown to be flexible in setting up successful transition 
scenarios when input parameters are slightly varied.
DYMOND is able to successfully set up a transition scenario, if the 
user ensures to set up a fuel management strategy correctly. 
It is important that \glspl{NFCSim} are flexible and resilient to 
perturbations in transition scenario input parameters. 
Despite 

% for sensitivity analysis 