\vspace{-0.2in}
The present United States nuclear fuel cycle faces challenges that hinder 
the expansion of nuclear energy technology. 
The U.S. Department of Energy identified four nuclear fuel cycle 
options we could transition to, which would overcome these challenges 
and make nuclear energy technology more desirable. 
In this work, our first goal is to model the transition from our current
state to a promising future end-state.
In reality, the real transition process inevitably diverges from the 
modeled scenario. 
Therefore, our second goal is to conduct sensitivity analysis 
studies to understand the nuances of the transition. 
To successfully analyze the transition from our current fuel cycle to 
a promising fuel cycle, we developed demand-driven deployment capabilities 
(\deploy) in Cyclus, a nuclear fuel cycle simulator. 
\deploy predictively and automatically deploys fuel cycle facilities 
to meet user-defined power demand.
We demonstrated \deploy's capability to automatically deploy fuel 
cycle facilities to set up a transition scenario from the current 
fleet to a combination of mixed oxide fuel pressurized water reactors 
and sodium fast-cooled reactors. 
In this work, we coupled nuclear fuel cycle simulators, DYMOND 
and \Cyclus, with Dakota, a sensitivity analysis tool. 
We demonstrated 
one-at-a-time, synergistic, and global sensitivity analysis studies.
We concluded that automated deployment of supporting fuel cycle 
facilities is crucial for conducting sensitivity analysis studies 
with nuclear fuel cycle simulators, to ensure that the simulation 
adapts to the new parameters by minimizing idle reactor capacity. 
We recommend that a comprehensive sensitivity analysis of a 
nuclear fuel cycle transition scenario begin with a global 
sensitivity analysis study to gain a general overview of the 
influential input variables for the output variables of interest. 
Using the global sensitivity analysis results, 
conduct one-at-a-time and synergistic sensitivity 
analysis to determine quantitative trends and impacts of influential 
input variables on specific output variables.
