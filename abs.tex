\vspace{-1cm}
The present United States nuclear fuel cycle faces challenges that hinder 
the expansion of nuclear energy technology. 
The U.S. Department of Energy identified four classes of nuclear fuel cycle 
options the U.S could transition to, which would overcome these challenges 
and make nuclear energy technology more desirable. 
The transitions have been modeled by various nuclear fuel cycle simulators
\cite{feng_standardized_2016,bae_standardized_2019}. 
However, most fuel cycle simulators require the user to define a deployment 
scheme for all supporting facilities to avoid any supply chain gaps, which becomes 
tedious for complex transition scenarios.
This work developed a capability in \Cyclus, a nuclear fuel cycle simulator, 
to automatically deploy fuel cycle 
facilities to meet a user-defined power demand. 
This work demonstrated \Cyclus' successful deployment of fuel cycle facilities
in a transition scenario from the current 
LWR fleet to a closed fuel cycle with continuous recycling of transuranics in fast and 
thermal spectrum reactors.
In reality, the technology transition process inevitably diverges from the 
modeled scenario. 
This work coupled the nuclear fuel cycle simulator tools, \Cyclus and DYMOND, 
with Dakota, a sensitivity analysis toolkit. 
This work used \Cyclus and DYMOND to conduct one-at-a-time, synergistic, 
and, global sensitivity analysis studies, to understand the impact changes 
in technology deployment strategies have on the transition.
This work compared \Cyclus' and DYMOND's sensitivity analysis capabilities 
and concluded that automated deployment of supporting fuel cycle 
facilities is crucial for conducting sensitivity analyses 
with nuclear fuel cycle simulators, to ensure that the simulation 
adapts to the new parameters by minimizing idle reactor capacity. 
The results demonstrated that time can be saved if a comprehensive 
sensitivity analysis of a nuclear fuel cycle transition scenario 
begins with a global sensitivity analysis study to gain a general 
overview of the influential input variables for the performance metrics. 
Then, based on the global sensitivity analysis results, limited 
one-at-a-time and synergistic sensitivity analyses are conducted 
to determine quantitative trends and impacts of influential 
input variables on the performance metrics.