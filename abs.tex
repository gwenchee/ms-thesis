The present United States nuclear fuel cycle faces challenges that hinder 
the expansion of nuclear energy technology. 
The U.S. Department of Energy identified four classes of nuclear fuel cycle 
options the U.S could transition to, which would overcome these challenges 
and make nuclear energy technology more desirable. 
The transitions have been modeled by various nuclear fuel cycle simulators. 
However, most fuel cycle simulators require the user to define a deployment 
scheme for all supporting facilities to avoid any supply chain gaps, which becomes 
tedious for complex transition scenarios.
This thesis developed a capability in \Cyclus, a nuclear fuel cycle simulator, 
to automatically deploy fuel cycle 
facilities to meet user-defined power demand. 
This new capability successfully deployed fuel cycle facilities
in a transition scenario from the current 
light water reactor fleet to a closed fuel cycle with continuous recycling of transuranics in fast and 
thermal reactors.
In reality, these transition scenarios inevitably diverge from the 
modeled scenario. 
This work coupled the nuclear fuel cycle simulator tools, \Cyclus and DYMOND, 
with Dakota, a sensitivity analysis toolkit. 
This work conducted one-at-a-time, synergistic, and 
global sensitivity analysis with \Cyclus-Dakota and DYMOND-Dakota,
to understand the interdependence of input parameters on the  
transition performance from the current 
light water reactor fleet to a closed fuel cycle in which transuranics are recycled to fuel 
mixed oxide fuel thermal reactors and sodium fast-cooled reactors. 
The global sensitivity analysis concluded that 
the transition year input parameter was the most influential
to the final depleted uranium and total idle reactor capacity 
performance metrics, and  
the fleet share ratio and cooling time input parameters 
were the most influential to the final high level waste amount in the 
simulation. 
The one-at-a-time sensitivity analysis showed that varying transition 
year from 80 to 84 years increased the final depleted uranium amount by 
1.13\% and reduced the total idle reactor capacity by 10.36\%. 
The one-at-a-time sensitivity analysis also showed that varying 
fleet share ratio 
(mixed oxide fuel light water reactor: sodium fast-cooled reactor) 
from 0:100 to 20:80 reduced the 
final high level waste amount by 2\%, and varying the used fuel cooling time from 0 to 
8 years reduced the final high level waste amount by 4\%. 
Therefore, an optimized transition scenario that minimizes final 
high level waste amount, final depleted uranium amount, and total idle capacity 
must have a fleet share ratio of 20:80, used fuel cooling time 
of 8 years, and a transition year at 83 years. 
This work compared \Cyclus-Dakota's and DYMOND-Dakota's sensitivity 
analysis capabilities 
and concluded that automated deployment of supporting fuel cycle 
facilities is crucial for conducting sensitivity analyses 
with nuclear fuel cycle simulators, to ensure that the simulation 
adapts to the new parameters by minimizing idle reactor capacity. 
The results demonstrated that time is saved if a comprehensive 
sensitivity analysis of a nuclear fuel cycle transition scenario 
begins with a global sensitivity analysis study to gain a general 
overview of the influential input variables for the performance metrics. 
Then, based on the global sensitivity analysis results, a reduced number of 
one-at-a-time and synergistic sensitivity analyses are conducted 
to determine quantitative trends and impacts of influential 
input variables on the performance metrics.